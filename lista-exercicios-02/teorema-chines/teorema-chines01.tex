Sobre o teorema chinês dos restos, responda:
\begin{enumerate}
	\item Se $p$ é um primo, qual o valor de $\Phi(p)$? Mostre que isso funciona para $p = 3, p = 5$. \\
	\emph{Resolução.} \\
	Tomando um $p$ primo, para todo $x < p$, temos que $mdc(x, p) = 1$. Portanto, temos que $\Phi(p) = p - 1$, pois é a quantidade de números menores que $p$. Podemos mostrar:
	\begin{itemize}
		\item $p = 3$: $\Phi(3) = \#\{1, 2\} = 2$;
		\item $p = 5$: $\Phi(5) = \#\{1, 2, 3, 4\} = 4$.
	\end{itemize}

	\item Considere o exemplo de representação de números por resíduos independentes, usando duas bases ($n = 2$): $m_1 = 3, m_2 = 5$. Calcule $M_1$ e $M_2$ conforme definidos acima, para este exemplo. \\
	\emph{Resolução.} \\
	Sejam $m_1, m_2, m_r \in \mathbb{Z}_+^*$, onde $m_1 = 3, m_2 = 5$. \\
	Temos que $m_1$ e $m_2$ são coprimos. Assim temos:
	\begin{displaymath}
		m_r = m_1 \cdot m_2 = 3 \cdot 5 = 15
	\end{displaymath}
	Podemos calcular $M_i$ tal que:
	\begin{itemize}
		\item $M_1 = (\frac{m_r}{m_1})^{\Phi(m_1)} = (\frac{15}{3})^2 = 25$. Note que $M_1$ é resíduo independente, pois $25 \equiv 1 \pmod{3}$ e $25 \equiv 0 \pmod{5}$.
		\item $M_2 = (\frac{m_r}{m_2})^{\Phi(m_2)} = (\frac{15}{5})^4 = 81$. Note que $M_2$ é resíduo independente, pois $81 \equiv 1 \pmod{5}$ e $81 \equiv 0 \pmod{3}$.
	\end{itemize}
\end{enumerate}
