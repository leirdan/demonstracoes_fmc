\documentclass[12pt]{article}
\usepackage[utf8]{inputenc}
\usepackage{amsmath}
\usepackage{amsfonts}
\usepackage{amssymb}
\newtheorem{definition}{Definição}
\newtheorem{postulate}{Postulado}
\newtheorem{theorem}{Teorema}
\newtheorem{corolary}{Corolário}
\newtheorem{principle}{Princípio}
\newtheorem{example}{Exemplo}
\newtheorem{note}{Nota}

\title{Lista de exercícios 02 - FMC}
\author{Andriel Vinicius de M. Fernandes}
\date{\today}

\begin{document}
\maketitle
\section{Congruência Modular}
\begin{enumerate}
	\item Demonstre: \\
Sejam $a, b, c, d, n \in \mathbb{Z}$, com $n > 1$. \\
Se $a\equiv c \pmod{n}$ e $b\equiv d \pmod{n}$, então:
\begin{enumerate}
	\item $(a \cdot b)\equiv (c \cdot d) \pmod{n}$; \\
	\emph{Resolução}. \\
	(1) Sejam $a, b, c, d, n \in \mathbb{Z}$, com $n > 1$, onde $a\equiv c \pmod{n}$ e $b\equiv d \pmod{n}$. \\
	(2) Por def., temos que $\exists k_1 \in \mathbb{Z}, k_1 \cdot n = a - c \implies a = k_1n + c$. \\
	(3) Por def., temos que $\exists k_2 \in \mathbb{Z}, k_2 \cdot n = b - d \implies b = k_2n + d$. \\
	Assim, temos:
	\begin{align*}
		a = k_1n + c &\quad \text{(Por aritmética)} \\
		ab = (k_1n + c) \cdot b &\quad \text{(Por multiplicação por $b$)} \\
		ab = bk_1n + bc &\quad \text{(Por distributividade em 5)} \\
		ab = (k_2n + d)k_1n + (k_2n + d)c &\quad \text{(Por substituição em $b$ por 3)} \\
		ab = k_1nk_2n + k_1nd + k_2nc + cd &\quad \text{(Por distributividade)} \\
		ab = n (k_1nk_2 + k_1d + k_2c) + cd &\quad \text{(Por evidência em $n$)} \\
		ab = nk_3 + cd &\quad \text{(Para $k_3 = (k_1nk_2 + k_1d + k_2c)$)} \\
		ab - cd = nk_3 &\quad \text{(Por aritmética)} \\
		\implies n | ab - cd &\quad \text{(Por def. de divisibilidade)} \\
	\end{align*}
	Portanto, $ab \equiv cd \pmod{n}$ pela def. de congruência.
	\item $a^m \equiv c^m \pmod{n}$, para qualquer $m \in \mathbb{Z}$. \\
	\emph{Resolução}. \\
	Vamos demonstrar por indução em $m$. \\
		(1) Suponha $a \equiv c \pmod{n}$. \\
		(2) Seja $P(m) := a^m \equiv c^m \pmod{n}$. \\
		(3) Passo base: $P(0)$.
			\begin{align*}
				P(0) :=& a^0 \equiv c^0 \pmod{n} \\
				&\implies 1 \equiv 1 \pmod{n} &\quad \text{(por $x^0 = 1$)}
			\end{align*}
		Por reflexividade, $P(0)$ é válido. \\
		(4) Hipótese Indutiva: seja um $k \in \mathbb{Z}$ arbitrário, tal que $P(k):= a^k \equiv c^k \pmod{n}$. Assim, $a = c + n \cdot w_0$, para $w_0 \in \mathbb{Z}$, por definição de congruência em (1). \\
		(5) Logo: 
		\begin{align*}
			a^k \cdot a &= (c + n \cdot w_0) \cdot a^k &\quad \text{(por reescrita)} \\ 
			&= c \cdot a^k + a^k \cdot n \cdot w_0 &\quad \text{(por distributividade)} \\
			&= c \cdot (c^k + n \cdot w_1) + (c^k + n \cdot w_1) \cdot (n \cdot w_0) &\quad \text{(por H.I)} \\
			&= cc^k + cnw_1 + c^kn + nw_1nw_0 &\quad \text{(por distributividade)} \\
			&= cc^k + n(cw_1 + c^k + w_0nw_1) &\quad \text{(por evidência)} \\
			&= cc^k + nw_2 &\quad \text{(para $w_2 = (cw_1 + c^k + w_0nw_1$))} \\
		\end{align*}
		Portanto, $a^{k+1} \equiv c^{k+1} \pmod{n}$.
\end{enumerate}



	\item Sejam $a, b, c, n \in \mathbb{Z}, n > 0$. \\
Demonstre que $ac \equiv bc \pmod{n} \iff a \equiv b \pmod{\frac{n}{mdc(c, n)}}$. \\
\emph{Resolução.} \\
Demonstremos primeiro que $ac \equiv bc \pmod{n} \implies a \equiv b \pmod{\frac{n}{mdc(c, n)}}$. \\
(1) Assuma $ac \equiv bc \pmod{n}$, por hipótese. \\
(2) Logo, $n | ac - bc$, por def. de congruência. \\
(3) Então $\exists k_1 \in \mathbb{Z}, k_1n = ac - bc$, por def. de divisibilidade em (2). \\
(4) Note que:
\begin{align*}
	k_1n = ac - bc &\implies k_1n = (a - b)c &\quad \text{(por aritmética)} \\
	&\implies \frac{k_1n}{mdc(c, n)} = \frac{(a-b)c}{mdc(c, n)} &\quad \text{(por div. de $mdc(c, n)$)} \\
\end{align*}
(5) Note que $mdc(\frac{n}{mdc(c, n)}, \frac{c}{mdc(c, n)}) = 1$. Ou seja, eles são coprimos. \\
(6) Por reescrita em (4), note que:
\begin{align*}
	\frac{n}{mdc(c, n)} | \frac{(a-b)c}{mdc(c,n)}
\end{align*}
(7) Contudo, por (5), temos que:
\begin{align*}
	\frac{n}{mdc(c, n)} | a - b \implies a \equiv b \pmod{\frac{n}{mdc(c, n)}} \quad \text{(por def. de congruência)}
\end{align*}

Agora, demonstremos que $a \equiv b \pmod{\frac{n}{mdc(c, n)}} \implies ac \equiv bc \pmod{n}$. \\
(1) Assuma $a \equiv b \pmod{\frac{n}{mdc(c, n)}}$, por hipótese. \\
(2) Logo, $frac{n}{mdc(c, n)}} | a - b$, por def. de congruência. \\
(3) Então, $\exists k_1 \in \mathbb{Z}, frac{n}{mdc(c, n)}} k_1 = a - b$, por def. de divisibilidade em (2). \\
(4) Seja $d = mdc(c, n)$. Então, $d|c$ e $d|n$, por def. de mdc. \\
(5) Assim, $\exists k_2, k_3 \in \mathbb{Z}, k_2d = c, k_3d = n$, por def. de divisibilidade em (4). \\
Temos:
\begin{align*}
	frac{n}{mdc(c, n)}} k_1 = a - b &\implies frac{n}{mdc(c, n)}}k_1c = (a - b)c &\quad \text{(por multiplicação por $c$)} \\ &\implies
	frac{n}{d}} k_1c = ac - bc &\quad \text{(por substituição com $d$)} \\ &\implies
	frac{n}{d}} k_1 k_2d = ac - bc &\quad \text{(por (5))} \\ &\implies
	n k_1 k_2 = ac - bc &\quad \text{(por aritmética)} \\ &\implies
	n k_4 = ac - bc &\quad \text{(para $k_4 = k_1k_2$)} \\ &\implies
	ac \equiv bc \pmod{n} &\quad \text{(por def. de congruência)} 
\end{align*}

Portanto, está demonstrado que $ac \equiv bc \pmod{n} \iff a \equiv b \pmod{\frac{n}{mdc(c, n)}}$. \\

	\item Demonstre que $mdc(2^a - 1, 2^b - 1) = 2^{mdc(a, b)} - 1$. \\
\emph{Resolução.} \\
Sejam $a, b \in \mathbb{Z}, a, b \ge 0$. Seja $d = mdc(a, b)$. \\
(Provemos que $2^d - 1 \mid 2^a - 1 \land 2^b - 1$.) \\
(1) Note que $2^d - 1 | 2^d - 1$. \\
(2) Ou seja, $2^d \equiv 1 \pmod{2^d - 1}$, por def. de congruência modular. \\
(3) Por potenciação, temos $(2^d)^\frac{a}{d} \equiv 1^\frac{a}{d} \pmod{2^d - 1}$. \\
(4) Por aritmética, temos $2^a \equiv 1 \pmod{2^d - 1}$. \\
(5) Por congruência, temos $2^d - 1 | 2^a - 1$. \\
(6) Analogamente, pelos passos (3), (4) e (5), temos $2^d - 1 | 2^b - 1$. \\
(7) Portanto, $2^d - 1 | 2^a - 1 \text{e} 2^b - 1$. \\

[Agora, mostremos que $2^d - 1$ é o melhor divisor destes números.] \\
A concluir.

	\item Quantas soluções inteiras existem para $x$, com $0 \leq x < 150$ para a congruência linear $63x \equiv 30 \pmod{150}$? Quais são elas? \\
\emph{Resolução.} \\
Verifiquemos o $mdc(150, 63)$:
\begin{align*}
mdc(150, 63): 
	150 &= 2 \cdot 63 + 24 &\quad (24 = 63 \cdot 2 - 150) \\
	63 &= 2 \cdot 24 + 15 &\quad (15 = 24 \cdot 2 - 63) \\
	24 &= 1 \cdot 15 + 9 &\quad (9 = 24 - 1 \cdot 15) \\
	15 &= 1 \cdot 9 + 6 &\quad (6 = 15 - 9) \\
	9 &= 3 \cdot 3 + 0 
\end{align*}
Portanto, $mdc(150, 63) = 3$. \\
Note que, pela regra do cancelamento geral, temos:
\begin{align*}
	63x \equiv 30 \pmod{150} \implies 21x * 3 \equiv 10 * 3 \pmod {150} \iff 21x \equiv 10 \pmod{\frac{150}{3}}
\end{align*}
Logo, temos que resolver $21x \equiv 10 \pmod{\frac{150}{3}}$. \\
Verifiquemos:
\begin{align*}
mdc(50, 21):
	50 &= 2 \cdot 21 + 8 &\quad (8 = 50 - 2 \cdot 21)
	21 &= 2 \cdot 8 + 5 &\quad (5 = 21 - 2 \cdot 8)
	8 &= 1 \cdot 5 + 3 &\quad (3 = 8 - 1 \cdot 5)
	5 &= 1 \cdot 3 + 2 &\quad (2 = 5 - 1 \cdot 3)
	3 &= 1 \cdot 2 + 1 &\quad (1 = 3 - 1 \cdot 2)
	2 &= 1 \cdot 2 + 0
\end{align*}
Portanto, $mdc(50, 21) = 1$. \\
Assim, vamos encontrar o inverso de 21 módulo 50: 
\begin{align*}
	1 &= 3 - 2 \\ 
	&= 3 - (5 - 3) \\
	&= 2 \cdot 3 - 5 \\
	&= 2 \cdot (8 - 5) - 5 \\
	&= 2 \cdot 8 - 3 \cdot 5 \\
	&= 2 \cdot 8 - 3 \cdot (21 - 2 \cdot 8) \\
	&= 2 \cdot 8 - 3 \cdot 21 + 6 \cdot 8 \\
	&= 8 \cdot 8 - 3 \cdot 21 \\
	&= 8 \cdot (50 - 2 \cdot 21) - 3 \cdot 21 \\
	&= 8 \cdot 50 - 16 \cdot 21 - 3 \cdot 21 \\
	&= 8 \cdot 50 - 19 \cdot 21
\end{align*}
Como $-19 \equiv 31 \pmod{50}$, temos que o inverso modular de 21 é 31. \\
Agora, note que:
\begin{align*}
	31 \cdot 21x \equiv 10 \cdot 31 \pmod{50} &\implies \\
	651x \equiv 310 \pmod{50} &\implies \\
	x \equiv 10 \pmod{50} \quad (651 \equiv 1 \pmod{50}) &\implies \\
	x = 50 \cdot t + 10 \quad \text{(Por def.)}
\end{align*}
para $0 \leq t \leq 2$. \\
Portanto, as 3 possíveis soluções para a congruência linear são $x = 10, x = 60, x= 110$.


	\item Qual o menor valor positivo que satisfaz esta congruência linear?
\begin{displaymath}
	81x \equiv 12 \pmod{264}
\end{displaymath}

	\item Calcule $(8^{10} - 128^{1796}) \pmod{13}$. Mostre todos os resultados intermediários. Durante o processo nenhum número com mais de 3 dígitos deve ser gerado. \\
\emph{Resolução.} \\
Note que:
\begin{displaymath}
	(8^{10} - 128^{1796}) \pmod{13} \implies (8^{10} \bmod{13}) - (128^{1796} \bmod{13}) \pmod{13}
\end{displaymath}

A princípio calculemos $(8^{10} \bmod{13})$. \\
Note que $8^{10} = 8^2 \cdot 8^4 \cdot 8^4$. Vamos calcular cada:
\begin{itemize}
	\item $8^2 \bmod{13} = 64 \bmod{13} = 12$ 
	\item $8^4 \bmod{13} = (8^2 \bmod{13})^2 \bmod{13} = 12^2 \bmod{13} = 1$ 
\end{itemize}
Portanto:
\begin{align*}
	8^{10} \bmod{13} &= (8^2 \bmod{13} \cdot 8^4 \bmod{13} \cdot 8^4 \bmod{13}) \bmod{13} \\
	&= (12 \cdot 1 \cdot 1) \bmod{13} \\
	&= 12
\end{align*}

Agora calculemos $(128^{1796}) \bmod{13}$. \\
Note que $(128 \bmod{13})^{1796} \bmod{13} = 11^{1796} \bmod{13}$. \\
Veja que $11^{1796} = 11^{1024} \cdot 11^{256} \cdot 11^{256} \cdot 11^{256} \cdot 11^4$.
Vamos calcular cada:
\begin{itemize}
	\item $11^4 \bmod{13} = (11^2 \bmod{13})^2 \bmod{13} = 4^2 \bmod{13} = 3$
	\item $11^{256} \bmod{13}$:
	\begin{itemize}
		\item Note que $256 = 21 \cdot 12 + 4$;
		\item Pelo Pequeno Teorema de Fermat, tomando o primo 13 e o inteiro 11, sabemos que $11^{12} \equiv 1 \pmod{13}$;
		\item Logo:
			\begin{align*}
				11^{256} &\equiv (11^{12})^{21} \cdot 11^4 \pmod{13} \\
					&\equiv 1^{21} \cdot 11^4 \pmod{13} \quad \text{[pois $11^{12} \equiv 1 \pmod{13}$]} \\
					&\equiv 11^4 \pmod{13}  \\
					&\equiv 3 \\
			\end{align*}
	\end{itemize}
	\item $11^{1024} \bmod{13} = (11^{256} \bmod{13})^4 \bmod{13} = 3^4 \bmod{13} = 3$
\end{itemize}

Desse modo, temos:
\begin{align*}
	11^{1796} \bmod{13} = (3 \cdot 3 \cdot 3 \cdot 3 \cdot 3) \bmod{13} = 243 \bmod{13} = 9
\end{align*}

Portanto, temos a operação final:
\begin{align*}
	(8^{10} - 128^{1796}) \bmod{13} &= (8^{10} \bmod{13}) - (128^{1796} \bmod{13}) \bmod{13} \\
	&= (12 - 9) \bmod{13} \\ 
	&= 3
\end{align*}

	\item Demonstre que para quaisquer inteiros $a, b, m_1, m,2$, com $m_1, m_2 > 0$, se $a \equiv b \pmod{mmc(m_1, m_2)}$ então $a \equiv b \pmod{m_1}$. \\
\emph{Resolução.}
Sejam $a, b, m_1, m,2 \in \mathbb{Z}$, com $m_1, m_2 > 0$. \\
(1) Suponha que $a \equiv b \pmod{mmc(m_1, m_2)}$, por hipótese. \\
(2) Temos $mmc(m_1, m_2) \mid a - b$, por def. de congruência. \\
(3) Temos $m_1 \mid mmc(m_1, m_2)$ por def. de mmc. \\
(4) Temos que, se $m_1 \mid mmc(m_1, m_2)$ e $mmc(m_1, m_2) \mid a - b$, então $m_1 \mid a - b$ por transitividade da div. \\
(5) Portanto, temos que $a \equiv b \pmod{m_1}$, por def. de congruência. \\

\end{enumerate}
\section{Teorema Chinês dos restos}
\begin{enumerate}
	\item Sobre o teorema chinês dos restos, responda:
\begin{enumerate}
	\item Se $p$ é um primo, qual o valor de $\Phi(p)$? Mostre que isso funciona para $p = 3, p = 5$. \\
	\emph{Resolução.} \\
	Tomando um $p$ primo, para todo $x < p$, temos que $mdc(x, p) = 1$. Portanto, temos que $\Phi(p) = p - 1$, pois é a quantidade de números menores que $p$. Podemos mostrar:
	\begin{itemize}
		\item $p = 3$: $\Phi(3) = \#\{1, 2\} = 2$;
		\item $p = 5$: $\Phi(5) = \#\{1, 2, 3, 4\} = 4$.
	\end{itemize}

	\item Considere o exemplo de representação de números por resíduos independentes, usando duas bases ($n = 2$): $m_1 = 3, m_2 = 5$. Calcule $M_1$ e $M_2$ conforme definidos acima, para este exemplo. \\
	\emph{Resolução.} \\
	Sejam $m_1, m_2, m_r \in \mathbb{Z}_+^*$, onde $m_1 = 3, m_2 = 5$. \\
	Temos que $m_1$ e $m_2$ são coprimos. Assim temos:
	\begin{displaymath}
		m_r = m_1 \cdot m_2 = 3 \cdot 5 = 15
	\end{displaymath}
	Podemos calcular $M_i$ tal que:
	\begin{itemize}
		\item $M_1 = (\frac{m_r}{m_1})^{\Phi(m_1)} = (\frac{15}{3})^2 = 25$. Note que $M_1$ é resíduo independente, pois $25 \equiv 1 \pmod{3}$ e $25 \equiv 0 \pmod{5}$.
		\item $M_2 = (\frac{m_r}{m_2})^{\Phi(m_2)} = (\frac{15}{5})^4 = 81$. Note que $M_2$ é resíduo independente, pois $81 \equiv 1 \pmod{5}$ e $81 \equiv 0 \pmod{3}$.
	\end{itemize}
\end{enumerate}

	\item Demonstre que, para $M_j = (\frac{m}{m_j})^{\Phi{m_j}}$, podemos concluir que $M_j \equiv 1 \pmod{m_j}$. \\
\emph{Resolução.} \\
Seja $m$ o produtório de todos os módulos de um sistema de congruências, ou seja, $m = m_1 \cdot m_2 \cdot \cdots \cdot m_j$, e $m_j \in \mathbb{Z}$. \\
Seja $M_j = (\frac{m}{m_j})^{\Phi{m_j}}$, com $M_j \in \mathbb{Z}$. \\
(1) Note que $m_j | m$, pois $\exists k \in \mathbb{Z}, k \cdot m_j = m$, onde $k = \frac{m}{m_j} = m_1 \cdot m_2 \cdot \cdots$. \\
(2) Então, temos que $mdc(k, m_j) = 1$, pois não há divisores em comum entre eles, visto que todos os módulos são coprimos entre si. \\
(3) Logo, temos que $k^{\Phi{m_j}} \equiv 1 \pmod{m_j}$, pelo teorema de Euler. \\
(4) Portanto, $M_j \equiv 1 \pmod{m_j}$ por substituição.

	\item Considere um computador que representa números inteiros de 0 a $2^{10} - 1$, ou seja, de 10 bits. Para aumentar a capacidade de processamento deste computador, podemos utilizar o teorema chinês dos restos com os módulos $2^{9} - 1, 2^{7} - 1, 2^{5} - 1$.
\begin{enumerate}
	\item Como representar $2^{11}, 2^{13}$ nos módulos acima? \\ 
	\emph{Resolução.} 
	\begin{itemize}
		\item $2^{11} \bmod{2^9 - 1} = 2^9 \cdot 2^2 \bmod{511} = 1 \cdot 4 \bmod{511} = 4$
		\item $2^{11}\bmod{2^7 - 1} = 2^7 \cdot 2^4 \bmod{127} = 1 \cdot 16 \bmod{127} = 16$
		\item $2^{11} \bmod{2^5 - 1} = 2^5 \cdot 2^6 \bmod{31} = 1 \cdot 64 \bmod{31} = 2$
		\item $2^{13} \bmod{2^9 - 1} = 2^9 \cdot 2^4 \bmod{511} = 1 \cdot 16 \bmod{511} = 16$
		\item $2^{13} \bmod{2^7 - 1} = 2^7 \cdot 2^6 \bmod{127} = 1 \cdot 64 \bmod{127} = 64$
		\item $2^{13} \bmod{2^5 - 1} = 2^5 \cdot 2^8 \bmod{31} = 1 \cdot 256 \bmod{31} = 8$
	\end{itemize}
	\item A partir das triplas de valores, determine a tripla da soma $2^{11} + 2^{13}$. \\ 
	\emph{Resolução.} Para $2^{11}$ temos $(4, 16, 2)$, e para $2^{13}$ temos $(16, 64, 8)$. Portanto, temos:
	\begin{align*}
		(4, 16, 1) + (16, 64, 8) &= (4 + 16 \bmod{511}, 16 + 64 \bmod{511}, 8 + 2 \bmod{31}) 
					\\ &= (20, 80, 10)
	\end{align*}
	\item Exiba o sistema de congruências da soma $2^{11} + 2^{13}$. \\
	\emph{Resolução.} A soma corresponde ao sistema abaixo:
	\begin{align*}
		x &\equiv 20 \pmod{511}
		x &\equiv 80 \pmod{127}
		x &\equiv 10 \pmod{31}
	\end{align*}
	\item Com o esquema referenciado nesta questão, o computador conseguiria representar números até qual valor? \\
	\emph{Resolução.} Até o produto dos módulos, sendo este $m = 511 \cdot 127 \cdot 31 = 2011807$.
\end{enumerate}

	\item Após muitos conflitos políticos no Brasil em 2030, os illuminatis decidem terceirizar uma intervenção militar e contratam um general chinês, que ficou encarregado de chefiar 500 soldados brasileiros antes de uma guerra civil, como para os chineses os ocidentais são todos muito parecidos, ele tinha uma certa dificuldade em contar os brasileiros. Seguindo uma intuição ancestral, após a guerra civil, o chinês alinhou os soldados em fileiras de 6 de forma que sobraram 3. Quando ele alinhou os soldados em fileiras de 7, também sobraram 3 soldados. Por fim, alinhou em fileiras de 11 e sobraram 5. Quantos soldados o general tinha no final? \\
\emph{Resolução}.
Temos o seguinte sistema de congruências:
\begin{align*}
	&s \equiv 3 \pmod{6} \\
	&s \equiv 3 \pmod{7} \\
	&s \equiv 5 \pmod{11} \\
\end{align*}

Pelo fato dos módulos serem coprimos, temos $m = m_1 \cdot m_2 \cdot m_3 = 6 \cdot 7 \cdot 11 = 462$. \\
Pelo teorema chinês dos restos, temos 
\begin{displaymath}
	s = s_1 \cdot M_1^{\Phi(m_1)} + s_2 \cdot M_2^{\Phi(m_2)} + s_3 \cdot M_3^{\Phi(m_3)} \pmod{m}
\end{displaymath}
onde $M_i = \frac{m}{m_1}$, $s_i = a_i \bmod{m_i}$ e $a_i$ é o resto em cada $m_i$. \\ 
Assim, temos os $s$:
\begin{itemize}
	\item $s_1 = 3 \bmod{6} = 3$;
	\item $s_2 = 3 \bmod{7} = 3$;
	\item $s_3 = 5 \bmod{11} = 5$.
\end{itemize}
e os $M$:
\begin{itemize}
	\item $M_1 = \frac{462}{6} = 77$;
	\item $M_2 = \frac{462}{7} = 66$;
	\item $M_3 = \frac{462}{11} = 42$.
\end{itemize}
Logo, temos:
\begin{align*}
	s &\equiv s_1 \cdot M_1^{\Phi(m_1)} + s_2 \cdot M_2^{\Phi(m_2)} + s_3 \cdot M_3^{\Phi(m_3)} \pmod{m} \\
	&\equiv 3 \cdot 77^2 + 3 \cdot 66^6 + 5 \cdot 42^{10} \pmod{462} \\
	&\equiv (3 \cdot 77^2 \bmod{462}) + (3 \cdot 66^6 \bmod{462}) + (5 \cdot 42^{10} \bmod{462}) \pmod{462} \\
	&\equiv (3 \cdot 77 \cdot 77 \bmod{462}) + (3 \cdot (66^2 \bmod{462})^3 \bmod{462}) + (5 \cdot 42^{10} \bmod{462}) \pmod{462} \\
	&= (3 \cdot 77 \cdot 77 \bmod{462}) + (3 \cdot (66^2 \bmod{462})^3 \bmod{462}) + \\& (5 \cdot 42^{10} \bmod{462}) \pmod{462} \\
\end{align*}
Resolvamos $(66^2 \bmod{462})^3$:
\begin{align*}
	(66^2 \bmod{462})^3 &= (4356 \bmod{462})^3 \bmod{462} \\
	&= 198^3 \bmod{462} \\
	&= (198 \bmod{462}) \cdot (198^2 \bmod{462}) \bmod{462} \\
	&= 198 \cdot (39204 \bmod{462}) \bmod{462} \\
	&= 198 \cdot 396 \bmod{462} \\
	&= 78408 \bmod{462} \\
	&= 330 \\
\end{align*}
Resolvamos também $(42^{10} \bmod{462})$. Note que $42^{10} = 42^2 \cdot 42^4 \cdot 42^4$. Vejamos cada um deles em módulo 462:
\begin{itemize}
	\item $(42^2 \bmod{462}) = 1764 \bmod{462} = 378$;
	\item $(42^4 \bmod{462}) = (42^2 \bmod{462})^2 \bmod{462} = 378^2 \bmod{462} = 142884 \bmod{462} = 126$;
\end{itemize}
Logo:
\begin{displaymath}
	42^{10} = (378 \cdot 126 \cdot 126) \bmod{462} = 6001128 \bmod{462} = 210
\end{displaymath}
Retomando:
\begin{align*}
	&(3 \cdot 77 \cdot 77 \bmod{462}) + (3 \cdot (66^2 \bmod{462})^3 \bmod{462}) + \\& (5 \cdot 42^{10} \bmod{462}) \pmod{462} \\ &\equiv (3 \cdot (77^2 \bmod{462}) \bmod{462}) + (3 \cdot 330 \bmod{462}) + (5 \cdot 210) \pmod{462} \\ 
	&\equiv (3 \cdot (5929 \bmod{462}) \bmod{462}) + (990 \bmod{462}) + (1050 \bmod{462}) \pmod{462} \\ 
	&\equiv (3 \cdot 385 \bmod{462}) + 66 + 126 \pmod{462} \\
	&\equiv 231 + 66 + 126 \pmod{462} \\
	&\equiv 423 \pmod{462}
\end{align*}

Portanto, o número final de soldados do general chinês é de 423.

\end{enumerate}
\section{Criptografia}
\begin{enumerate}
	\item Considere o sistema RSA com os seguintes parâmetros: 
$p = 3, q = 11, e = 17$. \\
\begin{enumerate}
	\item Determine as chaves pública e privada dos usuários. \\ 
	\emph{Resolução}. \\
	Vamos criar as chaves privada e pública de Alice e Beto. \\
	\begin{enumerate}
		\item Tome $p = 3, q = 11$. 
		\item Então, tome $n = 3 \cdot 11 = 33$. 
		\item Seja $\Phi$ o produto de $\phi(p)$ e $\phi(q)$. \\ 
			Ou seja, $\Phi = (3 - 1) \cdot (11 - 1) = 20$.
		\item Tome $e = 17$ tal que $mdc(17, \Phi) = 1$ e $1 < e < \Phi$. Verifiquemos:
			\begin{align*}
				mdc(20, 17): 20 &= 1 \cdot 17 + 3 &\quad (3 = 20 - 17) \\
						17 &= 5 \cdot 3 + 2 &\quad (2 = 17 - 5 \cdot 3) \\ 
						3 &= 1 \cdot 2 + 1 &\quad (1 = 3 - 2) \\ 
						2 &= 2 \cdot 1 0
			\end{align*}
		\item Vamos calcular o inverso modular de 17 módulo 20:
			\begin{align*}
				1 &= 3 - 2 \\ 
				&= 3 - (17 - 5 \cdot 3) \\
				&= 3 - (17 - 5 \cdot (20 - 17)) \\
				&= 3 - (17 - 5 \cdot 20 + 5 \cdot 17) \\
				&= 3 - 17 + 5 \cdot 20 - 5 \cdot 17 \\
				&= 20 - 17 - 17 + 5 \cdot 20 - 5 \cdot 17 \\
				&= 6 \cdot 20 - 7 \cdot 17 \\
			\end{align*}
		\item Como $-7 \equiv 13 \pmod{20}$, então temos o inverso modular $d = 13$. 
		\item Portanto, temos que as chaves pública e privada são, respectivamente, $(17, 33)$ e $13$.
	\end{enumerate}

	\item Desencripte o texto cifrado $c = 4$ para Alice. \\
	\emph{Resolução}. \\
	Tomando o texto cifrado $c = 4$ de Beto e a chave privada $d = 13$ de Alice, vamos desencriptar em um texto claro $m$ a partir da fórmula $m = c^d \bmod{n}$. \\
	Temos:
	\begin{align*}
		m &= 4^{13} \bmod{33} \\ 
		&= 4 \cdot 4^2 \cdot 4^2 \cdot 4^4 \cdot 4^4 \bmod{33}
	\end{align*}
	Vamos calcular cada potência em $\bmod{33}$:
	\begin{enumerate}
		\item $4 \bmod{33} = 4$;
		\item $4^2 \bmod{33} = 16 \bmod{33} = 16$;
		\item $4^4 \bmod{33} = (4^2 \bmod{33})^2 \bmod{33} = 16^2 \bmod{33} = 256 \bmod{33} = 25$.
	\end{enumerate}
	Ou seja,
	\begin{align*}
		m &= 4 \cdot 4^2 \cdot 4^2 \cdot 4^4 \cdot 4^4 \bmod{33} \\
		&= (4 \cdot 16 \cdot 16 \cdot 25 \cdot 25) \bmod{33} \\
		&= (4 \cdot 16^2 \cdot 25^2) \bmod{33} \\
		&= (1024 \cdot 625) \bmod{33} \\
		&= (1024 \bmod{33}) \cdot (625 \bmod{33}) \bmod{33} \\
		&= 1 \cdot 31 \bmod{33} \\
		&= 31 \\
	\end{align*}
	Portanto, o texto claro para Alice é $31$.
\end{enumerate}

\end{enumerate}
\end{document}

