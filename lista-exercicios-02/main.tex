\documentclass[12pt]{article}
\usepackage[utf8]{inputenc}
\usepackage{amsmath}
\usepackage{amsfonts}
\usepackage{amssymb}
\newtheorem{definition}{Definição}
\newtheorem{postulate}{Postulado}
\newtheorem{theorem}{Teorema}
\newtheorem{corolary}{Corolário}
\newtheorem{principle}{Princípio}
\newtheorem{example}{Exemplo}
\newtheorem{note}{Nota}

\title{Lista de exercícios 02 - FMC}
\author{Andriel Vinicius de M. Fernandes}
\date{\today}

\begin{document}
\maketitle
\section{Congruência Modular}
\begin{enumerate}
	\item Demonstre: \\
Sejam $a, b, c, d, n \in \mathbb{Z}$, com $n > 1$. \\
Se $a\equiv c \pmod{n}$ e $b\equiv d \pmod{n}$, então:
\begin{enumerate}
	\item $(a \cdot b)\equiv (c \cdot d) \pmod{n}$; \\
	\emph{Resolução}. \\
	(1) Sejam $a, b, c, d, n \in \mathbb{Z}$, com $n > 1$, onde $a\equiv c \pmod{n}$ e $b\equiv d \pmod{n}$. \\
	(2) Por def., temos que $\exists k_1 \in \mathbb{Z}, k_1 \cdot n = a - c \implies a = k_1n + c$. \\
	(3) Por def., temos que $\exists k_2 \in \mathbb{Z}, k_2 \cdot n = b - d \implies b = k_2n + d$. \\
	Assim, temos:
	\begin{align*}
		a = k_1n + c &\quad \text{(Por aritmética)} \\
		ab = (k_1n + c) \cdot b &\quad \text{(Por multiplicação por $b$)} \\
		ab = bk_1n + bc &\quad \text{(Por distributividade em 5)} \\
		ab = (k_2n + d)k_1n + (k_2n + d)c &\quad \text{(Por substituição em $b$ por 3)} \\
		ab = k_1nk_2n + k_1nd + k_2nc + cd &\quad \text{(Por distributividade)} \\
		ab = n (k_1nk_2 + k_1d + k_2c) + cd &\quad \text{(Por evidência em $n$)} \\
		ab = nk_3 + cd &\quad \text{(Para $k_3 = (k_1nk_2 + k_1d + k_2c)$)} \\
		ab - cd = nk_3 &\quad \text{(Por aritmética)} \\
		\implies n | ab - cd &\quad \text{(Por def. de divisibilidade)} \\
	\end{align*}
	Portanto, $ab \equiv cd \pmod{n}$ pela def. de congruência.
	\item $a^m \equiv c^m \pmod{n}$, para qualquer $m \in \mathbb{Z}$. \\
	\emph{Resolução}. \\
	Vamos demonstrar por indução em $m$. \\
		(1) Suponha $a \equiv c \pmod{n}$. \\
		(2) Seja $P(m) := a^m \equiv c^m \pmod{n}$. \\
		(3) Passo base: $P(0)$.
			\begin{align*}
				P(0) :=& a^0 \equiv c^0 \pmod{n} \\
				&\implies 1 \equiv 1 \pmod{n} &\quad \text{(por $x^0 = 1$)}
			\end{align*}
		Por reflexividade, $P(0)$ é válido. \\
		(4) Hipótese Indutiva: seja um $k \in \mathbb{Z}$ arbitrário, tal que $P(k):= a^k \equiv c^k \pmod{n}$. Assim, $a = c + n \cdot w_0$, para $w_0 \in \mathbb{Z}$, por definição de congruência em (1). \\
		(5) Logo: 
		\begin{align*}
			a^k \cdot a &= (c + n \cdot w_0) \cdot a^k &\quad \text{(por reescrita)} \\ 
			&= c \cdot a^k + a^k \cdot n \cdot w_0 &\quad \text{(por distributividade)} \\
			&= c \cdot (c^k + n \cdot w_1) + (c^k + n \cdot w_1) \cdot (n \cdot w_0) &\quad \text{(por H.I)} \\
			&= cc^k + cnw_1 + c^kn + nw_1nw_0 &\quad \text{(por distributividade)} \\
			&= cc^k + n(cw_1 + c^k + w_0nw_1) &\quad \text{(por evidência)} \\
			&= cc^k + nw_2 &\quad \text{(para $w_2 = (cw_1 + c^k + w_0nw_1$))} \\
		\end{align*}
		Portanto, $a^{k+1} \equiv c^{k+1} \pmod{n}$.
\end{enumerate}



	\item Quantas soluções inteiras existem para $x$, com $0 \leq x < 150$ para a congruência linear $63x \equiv 30 \pmod{150}$? Quais são elas? \\
\emph{Resolução.} \\
Verifiquemos o $mdc(150, 63)$:
\begin{align*}
mdc(150, 63): 
	150 &= 2 \cdot 63 + 24 &\quad (24 = 63 \cdot 2 - 150) \\
	63 &= 2 \cdot 24 + 15 &\quad (15 = 24 \cdot 2 - 63) \\
	24 &= 1 \cdot 15 + 9 &\quad (9 = 24 - 1 \cdot 15) \\
	15 &= 1 \cdot 9 + 6 &\quad (6 = 15 - 9) \\
	9 &= 3 \cdot 3 + 0 
\end{align*}
Portanto, $mdc(150, 63) = 3$. \\
Note que, pela regra do cancelamento geral, temos:
\begin{align*}
	63x \equiv 30 \pmod{150} &\implies 21x * 3 \equiv 10 * 3 \pmod {150} &\iff 21x \equiv 10 \pmod{\frac{150}{3}}
\end{align*}
Logo, temos que resolver $21x \equiv 10 \pmod{\frac{150}{3}}$. \\
Verifiquemos:
\begin{align*}
mdc(50, 21):
	50 &= 2 \cdot 21 + 8 &\quad (8 = 50 - 2 \cdot 21) \\
	21 &= 2 \cdot 8 + 5 &\quad (5 = 21 - 2 \cdot 8) \\
	8 &= 1 \cdot 5 + 3 &\quad (3 = 8 - 1 \cdot 5) \\
	5 &= 1 \cdot 3 + 2 &\quad (2 = 5 - 1 \cdot 3) \\
	3 &= 1 \cdot 2 + 1 &\quad (1 = 3 - 1 \cdot 2) \\
	2 &= 1 \cdot 2 + 0
\end{align*}
Portanto, $mdc(50, 21) = 1$. \\
Assim, vamos encontrar o inverso de 21 módulo 50: 
\begin{align*}
	1 &= 3 - 2 \\ 
	&= 3 - (5 - 3) \\
	&= 2 \cdot 3 - 5 \\
	&= 2 \cdot (8 - 5) - 5 \\
	&= 2 \cdot 8 - 3 \cdot 5 \\
	&= 2 \cdot 8 - 3 \cdot (21 - 2 \cdot 8) \\
	&= 2 \cdot 8 - 3 \cdot 21 + 6 \cdot 8 \\
	&= 8 \cdot 8 - 3 \cdot 21 \\
	&= 8 \cdot (50 - 2 \cdot 21) - 3 \cdot 21 \\
	&= 8 \cdot 50 - 16 \cdot 21 - 3 \cdot 21 \\
	&= 8 \cdot 50 - 19 \cdot 21
\end{align*}
Como $-19 \equiv 31 \pmod{50}$, temos que o inverso modular de 21 é 31. \\
Agora, note que:
\begin{align*}
	&31 \cdot 21x \equiv 10 \cdot 31 \pmod{50} \\ \implies 
	&651x \equiv 310 \pmod{50} \\ \implies 
	&x \equiv 10 \pmod{50} \quad \text{pois $(651 \equiv 1 \pmod{50})$} \\ &\implies
	&x = 50 \cdot t + 10 \quad \text{(Por def.)}
\end{align*}
para $0 \leq t \leq 2$. \\
Portanto, as 3 possíveis soluções para a congruência linear são $x = 10, x = 60, x= 110$.


	\item Qual o menor valor positivo que satisfaz esta congruência linear?
\begin{displaymath}
	81x \equiv 12 \pmod{264}
\end{displaymath}
\emph{Resolução}. \\
Verifiquemos:
\begin{align*}
	mdc(264, 81): 264 &= 3 \cdot 81 + 21 &\quad (21 = 264 - 3 \cdot 81) \\
	81 &= 3 \cdot 21 + 18 &\quad (18 = 81 - 3 \cdot 21) \\
	21 &= 1 \cdot 18 + 3 &\quad (3 = 21 - 18) \\
	18 &= 6 \cdot 3 + 0 
\end{align*}
Logo, $mdc(264, 81) = 3$. \\
Pela regra do cancelamento geral, temos que:
\begin{align*}
	81x \equiv 12 \pmod{264} \implies 27x \cdot 3 \equiv 4 \cdot 3 \pmod{264} \iff 27x \equiv 4 \pmod{\frac{264}{3}}
\end{align*}
Verifiquemos novamente o mdc entre 27 e 88:
\begin{align*}
	mdc(88, 27): 88 &= 3 \cdot 27 + 7 &\quad (7 = 88 - 3 \cdot 27) \\
	27 &= 3 \cdot 7 + 6 &\quad (6 = 27 - 3 \cdot 7) \\
	7 &= 1 \cdot 6 + 1 &\quad (1 = 7 - 6) \\
	6 &= 6 \cdot 1 + 0
\end{align*}
Logo, $mdc(88, 27) = 1$. \\
Assim, podemos calcular o inverso de 27 módulo 88 pelo algoritmo extendido de Euclides:
\begin{align*}
	1 &= 7 - 6 \\
	&= 7 - (27 - 3 \cdot 7)
	&= 7 - (27 - 3 \cdot (88 - 3 \cdot 27))
	&= 7 - (27 - 3 \cdot 88 + 9 \cdot 27)
	&= 88 - 3 \cdot 27 - (27 - 3 \cdot 88 + 9 \cdot 27)
	&= 4 \cdot 88 - 13 \cdot 27
\end{align*}
Como $-13 \equiv 75 \pmod{88}$, temos que o inverso modular de 27 módulo 88 é 75. \\
Agora, note que:
\begin{align*}
	&75 \cdot 27x \equiv 4 \cdot 75 \pmod{88} \\
	&\implies x \equiv 300 \pmod{88} \quad \text{[pois $27 \cdot 75 \equiv 1 \pmod{88}$]} \\
	&\implies x \equiv 300 \pmod{88} \\
	&\implies x \equiv 36 \pmod{88} \quad \text{[pois $300 \equiv 36 \pmod{88}$]} \\
\end{align*}
Logo, o menor inteiro que satisfaz a congruência é 36. \\

	\item Calcule $(8^{10} - 128^{1796}) \pmod{13}$. Mostre todos os resultados intermediários. Durante o processo nenhum número com mais de 3 dígitos deve ser gerado. \\
\emph{Resolução.} \\
Note que:
\begin{displaymath}
	(8^{10} - 128^{1796}) \pmod{13} \implies (8^{10} \bmod{13}) - (128^{1796} \bmod{13}) \pmod{13}
\end{displaymath}

A princípio calculemos $(8^{10} \bmod{13})$. \\
Note que $8^10 = 8^2 \cdot 8^4 \cdot 8^4$. Vamos calcular cada:
\begin{itemize}
	\item $8^2 \bmod{13} = 64 \bmod{13} = 12$ 
	\item $8^4 \bmod{13} = (8^2 \bmod{13})^2 \bmod{13} = 12^2 \bmod{13} = 1$ 
\end{itemize}
Portanto:
\begin{align*}
	8^{10} \bmod{13} &= (8^2 \bmod{13} \cdot 8^4 \bmod{13} \cdot 8^4 \bmod{13}) \bmod{13} \\
	&= (12 \cdot 1 \cdot 1) \bmod{13} \\
	&= 12
\end{align*}

Agora calculemos $(128^{1796}) \bmod{13}$. \\
Note que $(128 \bmod{13})^{1796} \bmod{13} = 11^{1796} \bmod{13}$. \\
Veja que $11^{1796} = 11^{1024} \cdot 11^{256} \cdot 11^{256} \cdot 11^{256} \cdot 11^4$.
Vamos calcular cada:
\begin{itemize}
	\item $11^4 \bmod{13} = (11^2 \bmod{13})^2 \bmod{13} = 4^2 \bmod{13} = 3$
	\item $11^{256} \bmod{13}$:
	\begin{itemize}
		\item Note que $256 = 21 \cdot 12 + 4$;
		\item Pelo Pequeno Teorema de Fermat, tomando o primo 13 e o inteiro 11, sabemos que $11^{12} \equiv 1 \pmod{13}$;
		\item Logo:
			\begin{align*}
				11^{256} &\equiv (11^{12})^{21} \cdot 11^4 \pmod{13}
					&\equiv 1^{21} \cdot 11^4 \pmod{13} \quad \text{[pois $11^{12} \equiv 1 \pmod{13}$]}
					&\equiv 11^4 \pmod{13} 
					&\equiv 3 
			\end{align*}
	\end{itemize}
	\item $11^{1024} \bmod{13} = (11^{256} \bmod{13})^4 \bmod{13} = 3^4 \bmod{13} = 3$
\end{itemize}

Desse modo, temos:
\begin{align*}
	11^{1796} \bmod{13} = (3 \cdot 3 \cdot 3 \cdot 3 \cdot 3) \bmod{13} = 243 \bmod{13} = 9
\end{align*}

Portanto, temos a operação final:
\begin{align*}
	(8^{10} - 128^{1796}) \bmod{13} &= (8^{10} \bmod{13}) - (128^{1796} \bmod{13}) \bmod{13} \\
	&= (12 - 9) \bmod{13} \\ 
	&= 3
\end{align*}

\end{enumerate}
\section{Teorema Chinês dos restos}
\begin{enumerate}
	\item Após muitos conflitos políticos no Brasil em 2030, os illuminatis decidem terceirizar uma intervenção militar e contratam um general chinês, que ficou encarregado de chefiar 500 soldados brasileiros antes de uma guerra civil, como para os chineses os ocidentais são todos muito parecidos, ele tinha uma certa dificuldade em contar os brasileiros. Seguindo uma intuição ancestral, após a guerra civil, o chinês alinhou os soldados em fileiras de 6 de forma que sobraram 3. Quando ele alinhou os soldados em fileiras de 7, também sobraram 3 soldados. Por fim, alinhou em fileiras de 11 e sobraram 5. Quantos soldados o general tinha no final? \\
\emph{Resolução}.
Temos o seguinte sistema de congruências:
\begin{align*}
	&s \equiv 3 \pmod{6} \\
	&s \equiv 3 \pmod{7} \\
	&s \equiv 5 \pmod{11} \\
\end{align*}

Pelos fato dos módulos serem coprimos, temos $m = m_1 \cdot m_2 \cdot m_3 = 6 \cdot 7 \cdot 11 = 462$. \\
Pelo teorema chinês dos restos, temos 
\begin{displaymath}
	s = s_1 \cdot M_1^{\Phi(m_1)} + s_2 \cdot M_2^{\Phi(m_2)} + s_3 \cdot M_3^{\Phi(m_3)} \pmod{m}
\end{displaymath}
onde $M_i = \frac{m}{m_1}$, $s_i = a_i \bmod{m_i}$ e $a_i$ é o resto em cada $m_i$. \\ 
Assim, temos os $s$:
\begin{itemize}
	\item $s_1 = 3 \bmod{6} = 3$;
	\item $s_2 = 3 \bmod{7} = 3$;
	\item $s_3 = 5 \bmod{11} = 5$.
\end{itemize}
e os $M$:
\begin{itemize}
	\item $M_1 = \frac{462}{6} = 77$;
	\item $M_2 = \frac{462}{7} = 66$;
	\item $M_3 = \frac{462}{11} = 42$.
\end{itemize}
Logo, temos:
\begin{align*}
	s &\equiv s_1 \cdot M_1^{\Phi(m_1)} + s_2 \cdot M_2^{\Phi(m_2)} + s_3 \cdot M_3^{\Phi(m_3)} \pmod{m} \\
	&\equiv 3 \cdot 77^2 + 3 \cdot 66^6 + 5 \cdot 42^{10} \pmod{462} \\
	&\equiv (3 \cdot 77^2 \bmod{462}) + (3 \cdot 66^6 \bmod{462}) + (5 \cdot 42^{10} \bmod{462}) \pmod{462} \\
	&\equiv (3 \cdot 77 \cdot 77 \bmod{462}) + (3 \cdot (66^2 \bmod{462})^3 \bmod{462}) + (5 \cdot 42^{10} \bmod{462}) \pmod{462} \\
	&= (3 \cdot 77 \cdot 77 \bmod{462}) + (3 \cdot (66^2 \bmod{462})^3 \bmod{462}) + \\& (5 \cdot 42^{10} \bmod{462}) \pmod{462} \\
\end{align*}
Resolvamos $(66^2 \bmod{462})^3$:
\begin{align*}
	(66^2 \bmod{462})^3 &= (4356 \bmod{462})^3 \bmod{462} \\
	&= 198^3 \bmod{462} \\
	&= (198 \bmod{462}) \cdot (198^2 \bmod{462}) \bmod{462} \\
	&= 198 \cdot (39204 \bmod{462}) \bmod{462} \\
	&= 198 \cdot 396 \bmod{462} \\
	&= 78408 \bmod{462} \\
	&= 330 \\
\end{align*}
Resolvamos também $(42^{10} \bmod{462})$. Note que $42^{10} = 42^2 \cdot 42^4 \cdot 42^4$. Vejamos cada um deles em módulo 462:
\begin{itemize}
	\item $(42^2 \bmod{462}) = 1764 \bmod{462} = 378$;
	\item $(42^4 \bmod{462}) = (42^2 \bmod{462})^2 \bmod{462} = 378^2 \bmod{462} = 142884 \bmod{462} = 126$;
\end{itemize}
Logo:
\begin{displaymath}
	42^{10} = (378 \cdot 126 \cdot 126) \bmod{462} = 6001128 \bmod{462} = 210
\end{displaymath}
Retomando:
\begin{align*}
	&(3 \cdot 77 \cdot 77 \bmod{462}) + (3 \cdot (66^2 \bmod{462})^3 \bmod{462}) + \\& (5 \cdot 42^{10} \bmod{462}) \pmod{462} \\ &\equiv (3 \cdot (77^2 \bmod{462}) \bmod{462}) + (3 \cdot 330 \bmod{462}) + (5 \cdot 210) \pmod{462} \\ 
	&\equiv (3 \cdot (5929 \bmod{462}) \bmod{462}) + (990 \bmod{462}) + (1050 \bmod{462}) \pmod{462} \\ 
	&\equiv (3 \cdot 385 \bmod{462}) + 66 + 126 \pmod{462} \\
	&\equiv 231 + 66 + 126 \pmod{462} \\
	&\equiv 423 \pmod{462}
\end{align*}

Portanto, o número final de soldados do general chinês é de 423.

\end{enumerate}
\section{Criptografia}
\begin{enumerate}
	\item Considere o sistema RSA com os seguintes parâmetros: 
$p = 3, q = 11, e = 17$. \\
\begin{enumerate}
	\item Determine as chaves pública e privada dos usuários. \\ 
	\emph{Resolução}. \\
	Vamos criar as chaves privada e pública de Alice e Beto. \\
	\begin{enumerate}
		\item Tome $p = 3, q = 11$. 
		\item Então, tome $n = 3 \cdot 11 = 33$. 
		\item Seja $\Phi$ o produto de $\phi(p)$ e $\phi(q)$. \\ 
			Ou seja, $\Phi = (3 - 1) \cdot (11 - 1) = 20$.
		\item Tome $e = 17$ tal que $mdc(17, \Phi) = 1$ e $1 < e < \Phi$. Verifiquemos:
			\begin{align*}
				mdc(20, 17): 20 &= 1 \cdot 17 + 3 &\quad (3 = 20 - 17) \\
						17 &= 5 \cdot 3 + 2 &\quad (2 = 17 - 5 \cdot 3) \\ 
						3 &= 1 \cdot 2 + 1 &\quad (1 = 3 - 2) \\ 
						2 &= 2 \cdot 1 0
			\end{align*}
		\item Vamos calcular o inverso de 17 módulo 20:

	\end{enumerate}

	\item Desencripte o texto cifrado $c = 4$ para Alice. \\
	\emph{Resolução}. \\
\end{enumerate}

\end{enumerate}
\end{document}

