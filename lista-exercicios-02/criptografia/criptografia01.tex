Considere o sistema RSA com os seguintes parâmetros: 
$p = 3, q = 11, e = 17$. \\
\begin{enumerate}
	\item Determine as chaves pública e privada dos usuários. \\ 
	\emph{Resolução}. \\
	Vamos criar as chaves privada e pública de Alice e Beto. \\
	\begin{enumerate}
		\item Tome $p = 3, q = 11$. 
		\item Então, tome $n = 3 \cdot 11 = 33$. 
		\item Seja $\Phi$ o produto de $\phi(p)$ e $\phi(q)$. \\ 
			Ou seja, $\Phi = (3 - 1) \cdot (11 - 1) = 20$.
		\item Tome $e = 17$ tal que $mdc(17, \Phi) = 1$ e $1 < e < \Phi$. Verifiquemos:
			\begin{align*}
				mdc(20, 17): 20 &= 1 \cdot 17 + 3 &\quad (3 = 20 - 17) \\
						17 &= 5 \cdot 3 + 2 &\quad (2 = 17 - 5 \cdot 3) \\ 
						3 &= 1 \cdot 2 + 1 &\quad (1 = 3 - 2) \\ 
						2 &= 2 \cdot 1 0
			\end{align*}
		\item Vamos calcular o inverso modular de 17 módulo 20:
			\begin{align*}
				1 &= 3 - 2 \\ 
				&= 3 - (17 - 5 \cdot 3) \\
				&= 3 - (17 - 5 \cdot (20 - 17)) \\
				&= 3 - (17 - 5 \cdot 20 + 5 \cdot 17) \\
				&= 3 - 17 + 5 \cdot 20 - 5 \cdot 17 \\
				&= 20 - 17 - 17 + 5 \cdot 20 - 5 \cdot 17 \\
				&= 6 \cdot 20 - 7 \cdot 17 \\
			\end{align*}
		\item Como $-7 \equiv 13 \pmod{20}$, então temos o inverso modular $d = 13$. 
		\item Portanto, temos que as chaves pública e privada são, respectivamente, $(17, 33)$ e $13$.
	\end{enumerate}

	\item Desencripte o texto cifrado $c = 4$ para Alice. \\
	\emph{Resolução}. \\
	Tomando o texto cifrado $c = 4$ de Beto e a chave privada $d = 13$ de Alice, vamos desencriptar em um texto claro $m$ a partir da fórmula $m = c^d \bmod{n}$. \\
	Temos:
	\begin{align*}
		m &= 4^{13} \bmod{33} \\ 
		&= 4 \cdot 4^2 \cdot 4^2 \cdot 4^4 \cdot 4^4 \bmod{33}
	\end{align*}
	Vamos calcular cada potência em $\bmod{33}$:
	\begin{enumerate}
		\item $4 \bmod{33} = 4$;
		\item $4^2 \bmod{33} = 16 \bmod{33} = 16$;
		\item $4^4 \bmod{33} = (4^2 \bmod{33})^2 \bmod{33} = 16^2 \bmod{33} = 256 \bmod{33} = 25$.
	\end{enumerate}
	Ou seja,
	\begin{align*}
		m &= 4 \cdot 4^2 \cdot 4^2 \cdot 4^4 \cdot 4^4 \bmod{33} \\
		&= (4 \cdot 16 \cdot 16 \cdot 25 \cdot 25) \bmod{33} \\
		&= (4 \cdot 16^2 \cdot 25^2) \bmod{33} \\
		&= (1024 \cdot 625) \bmod{33} \\
		&= (1024 \bmod{33}) \cdot (625 \bmod{33}) \bmod{33} \\
		&= 1 \cdot 31 \bmod{33} \\
		&= 31 \\
	\end{align*}
	Portanto, o texto claro para Alice é $31$.
\end{enumerate}
