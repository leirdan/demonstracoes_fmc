Considere o sistema RSA com os seguintes parâmetros: 
$p = 3, q = 11, e = 17$. \\
\begin{enumerate}
	\item Determine as chaves pública e privada dos usuários. \\ 
	\emph{Resolução}. \\
	Vamos criar as chaves privada e pública de Alice e Beto. \\
	\begin{enumerate}
		\item Tome $p = 3, q = 11$. 
		\item Então, tome $n = 3 \cdot 11 = 33$. 
		\item Seja $\Phi$ o produto de $\phi(p)$ e $\phi(q)$. \\ 
			Ou seja, $\Phi = (3 - 1) \cdot (11 - 1) = 20$.
		\item Tome $e = 17$ tal que $mdc(17, \Phi) = 1$ e $1 < e < \Phi$. Verifiquemos:
			\begin{align*}
				mdc(20, 17): 20 &= 1 \cdot 17 + 3 &\quad (3 = 20 - 17) \\
						17 &= 5 \cdot 3 + 2 &\quad (2 = 17 - 5 \cdot 3) \\ 
						3 &= 1 \cdot 2 + 1 &\quad (1 = 3 - 2) \\ 
						2 &= 2 \cdot 1 0
			\end{align*}
		\item Vamos calcular o inverso de 17 módulo 20:

	\end{enumerate}

	\item Desencripte o texto cifrado $c = 4$ para Alice. \\
	\emph{Resolução}. \\
\end{enumerate}
