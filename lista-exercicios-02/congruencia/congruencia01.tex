Demonstre: \\
Sejam $a, b, c, d, n \in \mathbb{Z}$, com $n > 1$. \\
Se $a\equiv c \pmod{n}$ e $b\equiv d \pmod{n}$, então:
\begin{enumerate}
	\item $(a \cdot b)\equiv (c \cdot d) \pmod{n}$; \\
	\emph{Resolução}. \\
	(1) Sejam $a, b, c, d, n \in \mathbb{Z}$, com $n > 1$, onde $a\equiv c \pmod{n}$ e $b\equiv d \pmod{n}$. \\
	(2) Por def., temos que $\exists k_1 \in \mathbb{Z}, k_1 \cdot n = a - c \implies a = k_1n + c$. \\
	(3) Por def., temos que $\exists k_2 \in \mathbb{Z}, k_2 \cdot n = b - d \implies b = k_2n + d$. \\
	Assim, temos:
	\begin{align*}
		&\text{(4)} \quad a = k_1n + c &\quad \text{(Por aritmética)} \\
		&\text{(5)} \quad ab = (k_1n + c) \cdot b &\quad \text{(Por multiplicação por $b$)} \\
		&\text{(6)} \quad ab = bk_1n + bc &\quad \text{(Por distributividade em 5)} \\
		&\text{(7)} \quad ab = (k_2n + d)k_1n + (k_2n + d)c &\quad \text{(Por substituição em $b$ por 3)} \\
		&\text{(8)} \quad ab = k_1nk_2n + k_1nd + k_2nc + cd &\quad \text{(Por distributividade)} \\
		&\text{(9)} \quad ab = n (k_1nk_2 + k_1d + k_2c) + cd &\quad \text{(Por evidência em $n$)} \\
		&\text{(10)} \quad ab = nk_3 + cd &\quad \text{(Para $k_3 = (k_1nk_2 + k_1d + k_2c)$)} \\
		&\text{(11)} \quad ab - cd = nk_3 &\quad \text{(Por aritmética)} \\
		&\text{(11)} \quad n | ab - cd &\quad \text{(Por def. de divisibilidade)} \\
	\end{align*}
	Portanto, $ab \equiv cd \pmod{n}$ pela def. de congruência.
	\item $a^m \equiv c^m \pmod{n}$, para qualquer $m \in \mathbb{Z}$. \\
	\emph{Resolução}. \\
	Vamos demonstrar por indução em $m$. \\
		(1) Suponha $a \equiv c \pmod{n}$. \\
		(2) Seja $P(m) := a^m \equiv c^m \pmod{n}$. \\
		(3) Passo base: $P(0)$.
			\begin{align*}
				P(0) :=& a^0 \equiv c^0 \pmod{n} \\
				&\implies 1 \equiv 1 \pmod{n} &\quad \text{(por $x^0 = 1$)}
			\end{align*}
		Por reflexividade, $P(0)$ é válido. \\
		(4) Hipótese Indutiva: seja um $k \in \mathbb{Z}$ arbitrário, tal que $P(k):= a^k \equiv c^k \pmod{n}$. Assim, $a = c + n \cdot w_0$, para $w_0 \in \mathbb{Z}$, por definição de congruência em (1). \\
		(5) Logo: 
		\begin{align*}
			&\text{(6)} &\quad a^k \cdot a = (c + n \cdot w_0) \cdot a^k &\quad \text{(por reescrita)} \\ 
			&\text{(7)} \quad &= c \cdot a^k + a^k \cdot n \cdot w_0 &\quad \text{(por distributividade)} \\
			&\text{(8)} \quad &= c \cdot (c^k + n \cdot w_1) + (c^k + n \cdot w_1) \cdot (n \cdot w_0) &\quad \text{(por H.I)} \\
			&\text{(9)} \quad &= cc^k + cnw_1 + c^kn + nw_1nw_0 &\quad \text{(por distributividade)} \\
			&\text{(10)} \quad &= cc^k + n(cw_1 + c^k + w_0nw_1) &\quad \text{(por evidência)} \\
			&\text{(11)} \quad &= cc^k + nw_2 &\quad \text{(para $w_2 = (cw_1 + c^k + w_0nw_1$))} \\
		\end{align*}
		Portanto, $a^{k+1} \equiv c^{k+1} \pmod{n}$.
\end{enumerate}


