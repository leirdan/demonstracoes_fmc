Demonstre o Lema de Euclides: sejam $a, b, p \in \mathbb{Z}$ com $p$ primo; se $p | ab$, então $p | a$ ou $p | b$ (note que é possível que $p$ divida tanto $a$ quanto $b$). \\
\emph{Demonstração.} \\
Sejam $a, b, p \in \mathbb{Z}$, onde $p$ é primo. \\
\textbf{(1)} Assuma $p | ab$. \\
Suponha $p \nmid a$. Logo, \textbf{(2)} $mdc(a, p) = 1$. \\
\textbf{(3)} Note que $\exists s, t \in \mathbb{Z}; mdc(a, p) = sa + tp$, pelo Teorema de Bezout. \\
\textbf{(4)} Note que $sa + tp = 1$, por \textbf{(3)} e \textbf{(2)}. \\
\textbf{(5)} Note também que $p | p$, logo $p | p \cdot (bt)$ por propriedade de divisibilidade. \\
\textbf{(6)} Note também que $p | ab$, logo $p | ab \cdot (s)$ por propriedade de divisibilidade. \\
\textbf{(7)} Dessa forma, temos que
\begin{align*}
    p | ((p \cdot bt) + (ab \cdot s)) & = p | b \cdot (sa + tp) & \quad \text{(Por aritmética)}   \\
                                      & = p | b                 & \quad \text{(Por \textbf{(4)})}
\end{align*}
\textbf{(8)} Portanto, $p | b$ e o lema é válido.