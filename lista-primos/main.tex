\documentclass[12pt]{article}
\usepackage[utf8]{inputenc}
\usepackage[portuguese]{babel}
\usepackage{amsmath}
\usepackage{amsfonts}
\usepackage{amssymb}
\newtheorem{definition}{Definição}
\newtheorem{postulate}{Postulado}
\newtheorem{theorem}{Teorema}
\newtheorem{corolary}{Corolário}
\newtheorem{principle}{Princípio}
\newtheorem{example}{Exemplo}
\newtheorem{note}{Nota}

\title{Lista de exercícios 01 - FMC}
\author{Andriel Vinicius de M. Fernandes}
\date{\today}

\begin{document}
\maketitle
\begin{theorem}
	Teorema Fundamental da Aritmética:
	Todo inteiro positivo $n$ pode ser escrito de maneira única como o produto de números primos, onde os fatores primos são escritos em ordem crescente de grandeza.
\end{theorem}
\section{Primos}
\begin{enumerate}
	\item Mostre que se $n$ é um inteiro composto qualquer ele possui um divisor primo menor ou igual a $\sqrt{n}$. \\
\emph{Demonstração.} \\
Seja $n$ um inteiro composto arbitrário. \\
Logo, \textbf{(1)} pelo Teorema Fundamental da Aritmética, $\exists a, b \in \mathbb{Z}, n = a\cdot b$, com $a, b > 1$. \\
\textbf{(2)} Note que, aplicando a definição de divisibilidade em \textbf{(1)}, temos que $a | n$. \\
\textbf{(3)} Suponha que $a \leq b$. \\
Note que:
\begin{align*}
    a \leq b                 & \implies a \leq \frac{n}{a}   & \quad \text{(Para $b = \frac{n}{a}$)} \\ &\implies
    a^2 \leq n               & \quad \text{(Por aritmética)}                                         \\ &\implies
    \sqrt{a^2} \leq \sqrt{n} & \quad \text{(Por aritmética)}                                         \\ &\implies
    a \leq \sqrt{n}          & \quad \text{(Por aritmética)}                                         \\
\end{align*}
\textbf{(4)} Logo, $a \leq \sqrt{n}$.\\
Assim, temos dois casos para $a$:
\begin{itemize}
    \item $a$ é primo. Nesse caso, $a$ não é decomposto em outros fatores e é imediato que $a$ é divisor primo de $n$ tal que $a \leq \sqrt{n}$ por \textbf{(4)} e \textbf{(2)}.
    \item $a$ é composto. Nesse caso, pelo Teorema Fundamental da Aritmética, temos que $a$ é resultado de um produto de números primos, onde um desses é um inteiro primo $p$. Aplicando a def. de divisibilidade, temos que $\exists p_0 \in \mathbb{Z}, p \cdot p_0 = a \iff p | a$. Assim, se $p | a$ e $a | n$ por \textbf{(2)}, então $p | n$ pela transitividade da divisão. Note ainda que $p \leq a \leq \sqrt{n} \implies p \leq \sqrt{n}$ por \textbf{(4)}.
\end{itemize}
\textbf{(5)} Portanto, está mostrado que existe sempre um divisor primo de um natural qualquer menor ou igual à raiz quadrada deste natural.
	\item Demonstre o Lema de Euclides: sejam $a, b, p \in \mathbb{Z}$ com $p$ primo; se $p | ab$, então $p | a$ ou $p | b$ (note que é possível que $p$ divida tanto $a$ quanto $b$). \\
\emph{Demonstração.} \\
Sejam $a, b, p \in \mathbb{Z}$, onde $p$ é primo. \\
\textbf{(1)} Assuma $p | ab$. \\
Suponha $p \nmid a$. Logo, \textbf{(2)} $mdc(a, p) = 1$. \\
\textbf{(3)} Note que $\exists s, t \in \mathbb{Z}; mdc(a, p) = sa + tp$, pelo Teorema de Bezout. \\
\textbf{(4)} Note que $sa + tp = 1$, por \textbf{(3)} e \textbf{(2)}. \\
\textbf{(5)} Note também que $p | p$, logo $p | p \cdot (bt)$ por propriedade de divisibilidade. \\
\textbf{(6)} Note também que $p | ab$, logo $p | ab \cdot (s)$ por propriedade de divisibilidade. \\
\textbf{(7)} Dessa forma, temos que
\begin{align*}
    p | ((p \cdot bt) + (ab \cdot s)) & = p | b \cdot (sa + tp) & \quad \text{(Por aritmética)}   \\
                                      & = p | b                 & \quad \text{(Por \textbf{(4)})}
\end{align*}
\textbf{(8)} Portanto, $p | b$ e o lema é válido.
	\item Demonstre que para qualquer inteiro $n$, se $n mod 7 = 2$, então $n^2 mod 7 = 2$. \\
\emph{Resolução.} \\
Tome $n = 9$. Note que
\begin{align*}
	9\mod 7 & = 9 - ((\frac{9}{7}) \cdot 7) & \quad \text{(Por def. de mod)} \\
	        & = 9 - 7                       & \quad \text{(Por aritmética})  \\
	        & = 2
\end{align*}
Mas, tomando $n^2 = 9^2 = 81$, temos:
\begin{align*}
	81\mod 7 & = 81 - ((\frac{81}{7} \cdot 7) & \quad \text{(Por def. de mod)} \\
	         & = 81 - 77                      & \quad \text{(Por aritmética})  \\
	         & = 4
\end{align*}
Portanto, a proposição é falsa.

\end{enumerate}
\end{document}

