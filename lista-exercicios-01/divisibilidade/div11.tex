Demonstre que $\forall n \in \mathbb{N}, 2 | (n^2 + n)$. \\
\emph{Resolução.} \\
Aplicaremos indução sobre $n$. \\
Seja $P(n) := \exists k \in \mathbb{N}, k \cdot 2 = n^2 + n$. \\
Passo base: $n = 0$
\begin{align*}
    2 | (0^2 + 0)
\end{align*}
Note que $2 | 0$ pois, $\forall n \in \mathbb{N}, n | 0$. Logo, $P(0)$ é verdadeiro. \\
Passo indutivo: \\
Seja um certo $k \in \mathbb{N}$. Temos:
\begin{displaymath}
    \exists w_1 \in \mathbb{N}, w_1 \cdot 2 = (k^2 + k)
\end{displaymath}
(Provemos que $P(k + 1) = \exists w_2 \in \mathbb{N}$ tal que $w_2 \cdot 2 = (k + 1)^2 + (k + 1)$).
Logo:
\begin{align*}
    P(k + 1) & = (k + 1)^2 + (k + 1)                                                \\
             & = k^2 + 2k + 1 + k + 1   & \quad \text{(Por aritmética)}             \\
             & = k^2 + k + 2 + 2k       & \quad \text{(Por aritmética)}             \\
             & = (w_1 \cdot 2) + 2 + 2k & \quad \text{(Por HI)}                     \\
             & = 2 \cdot (w_1 + 1 + k)  & \quad \text{(Por aritmética)}             \\
             & = 2 \cdot w_2            & \quad \text{(Para $w_2 = (w_1 + 1 + k)$)} \\
\end{align*}
Logo, $\exists w_2 \in \mathbb{N}$ (naturais são fechados para soma) tal que $w_2 \cdot 2 = (k + 1)^2 + (k + 1)$. \\
Portanto, o teorema é válido.