$\forall a, b, c \in \mathbb{N}$ Se $a | b$ e $b | c$, então $a | (b \cdot x + c \cdot y)$ para $\forall x, y \in \mathbb{N}$. \\
\emph{Resolução.} \\
Seja $a, b, c, x, y \in \mathbb{N}$, com $a > 0, b > 0$. \\
Temos que $a | b$, ou seja, $\exists w_1 \in \mathbb{N}, a \cdot w_1 = b$. \\
Temos que $b | c$, ou seja, $\exists w_2 \in \mathbb{N}, b \cdot w_1 = c$. \\
Note que, por aritmética:
\begin{align*}
    a \cdot w_1 = b \iff a \cdot w_1 \cdot x = b \cdot x 
\end{align*}
Analogamente, temos:
\begin{align*}
    b \cdot w_2 = c \iff b \cdot w_2 \cdot y = c \cdot y 
\end{align*}
Somando as duas equações, teremos:
\begin{align*}
    (a \cdot w_1 \cdot x) + (b \cdot w_2 \cdot y) &= (b \cdot x) + (c \cdot y) \\ &\implies
    (a \cdot w_1 \cdot x) + (a \cdot w_1 \cdot w_2 \cdot y) = (b \cdot x) + (c \cdot y) \\ &\quad \text{(Por reescrita de $b$)}\\ &\implies
    a \cdot (\cdot w_1 \cdot x + w_1 \cdot w_2 \cdot y) = (b \cdot x) + (c \cdot y) \\ &\quad \text{(Por aritmética)} \\ &\implies
    a \cdot w_3 = (b \cdot x) + (c \cdot y) \\ &\quad \text{(Para $w_3 = \cdot w_1 \cdot x + w_1 \cdot w_2 \cdot y$)}
\end{align*}
Logo, $\exists w_3 \cdot a = (b \cdot x) + (c \cdot y)$. 
Portanto, a proposição é verdadeira.