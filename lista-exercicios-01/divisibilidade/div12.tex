Demonstre que $\forall n \in \mathbb{N}, 6 | (n^3 - n)$.\\
\emph{Resolução.}\\
Aplicando indução sobre $n$. \\
Seja $P(n) := \exists k \in \mathbb{N}, k \cdot 6 = (n^3 - n)$ \\
Passo base: $n = 0$ \\
Note que $6 | (0^3 - 0)$ pois $\forall n \in \mathbb{N}, n | 0$. \\
Passo indutivo: \\
Seja um certo $k \in \mathbb{N}$. Temos:
\begin{displaymath}
    \exists w_1 \cdot 6 = (k^3 - k) \quad \text{(Hipótese Indutiva)}
\end{displaymath}
(Provemos que $P(k + 1) := \exists w_2 \in \mathbb{N}, w_2 \cdot 6 = (k + 1)^3 - (k + 1)$) \\
Logo:
\begin{align*}
    (k + 1)^3 - (k + 1) & = k^3 + 3k^2 + 3k + 1 - k - 1                  & \quad \text{(Por aritmética)}                                  \\
                        & = k^3 - k + 3k^2 + 3k                          & \quad \text{(Por aritmética)}                                  \\
                        & = (w_1 \cdot 6) + 3k^2 + 3k                    & \quad \text{(Por H.I)}                                         \\
                        & = 6 \cdot (w_1 + \frac{1}{2}k^2 + \frac{1}{2}) & \quad \text{(Por aritmética)}                                  \\
                        & = 6 \cdot w_2                                  & \quad \text{(Para $w_2 = w_1 + \frac{1}{2}k^2 + \frac{1}{2}$)} \\
\end{align*}
Logo, $\exists w_2 \in \mathbb{N}$ (naturais são fechados para soma e multiplicação) tal que $w_2 \cdot 6 = (k + 1)^3 - (k + 1)$. \\
Portanto, o teorema é válido.