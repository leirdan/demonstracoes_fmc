Calcule a soma dos $n$ primeiros pares, em seguida, demonstre por indução. \\
\emph{Resolução.}
Note que:
\begin{align*}
	0             & = 0  \\
	0 + 2         & = 2  \\
	0 + 2 + 4     & = 6  \\
	0 + 2 + 4 + 6 & = 12 \\
\end{align*}
Ou seja, a soma dos $n$ primeiros pares é dada por $n^2 + n$. Provemos essa proposição por indução sobre $n$. \\
Seja $P(n):= \sum\limits_{i = 0}^{n}2i = 0 + 2 + 4 + 6 + \cdots + n = n^2 + n$ \\
Passo base: $n = 0$
\begin{align*}
	\sum\limits_{i = 0}^{0} & = 0       & \quad \text{(Por def. base de $\sum$)} \\
	                        & = 0 + 0   & \quad(\text{Por aritmética})           \\
	                        & = 0^2 + 0 & \quad(\text{Por aritmética})           \\
	                        & = n^2 + n & \quad(\text{Para $n = 0$})
\end{align*}
Logo, $P(0)$ é verdadeiro. \\
Passo indutivo: \\
Seja um certo $k \in \mathbb{N}$ tal que
\begin{align*}
	P(k):= \sum\limits_{i = 0}^{k}2i = k^2 + k \quad \text{(Hipótese Indutiva)}
\end{align*}
(Provemos que $P(k + 1) = (k + 1)^2 + (k + 1)$). \\
Logo:
\begin{align*}
	\sum\limits_{i=0}^{k + 1}2i & = (\sum\limits_{i = 0}^{k}2i) + 2(k + 1) & \quad(\text{Por def. rec. de $\sum$}) \\
	                            & = (k^2 + k) + 2(k + 1)                   & \quad(\text{Por H.I})                 \\
	                            & = k^2 + k + 2k + 2                       & \quad(\text{Por aritmética})          \\
	                            & = k^2 + 3k + 2                           & \quad(\text{Por aritmética})          \\
	                            & = (k + 1)^2 + (k + 1)                    & \quad(\text{Por aritmética})
\end{align*}
Logo, $P(k + 1)$ é verdadeiro.
Portanto, a proposição é verdadeira.
