Calcule a soma dos $n$ primeiros pares, em seguida, demonstre por indução. \\
\emph{Resolução.}
Note que:
\begin{align*}
	0 &= 0 \\
	0 + 2 &= 2 \\
	0 + 2 + 4 &= 6 \\
	0 + 2 + 4 + 6 &= 12 \\
\end{align*}
Ou seja, a soma dos $n$ primeiros pares é dada por $n^2 + n$. Provemos essa proposição por indução. \\
\begin{enumerate}
	\item Seja $P(n):= \sum\limits_{i = 0}^{n}i = 0 + 2 + 4 + 6 + \cdots + 2n = n^2 + n$ \\
	\item Aplicaremos indução sobre $n \in \mathbb{N}$ \\
	\item Vejamos o passo-base $P(0)$: 
		\begin{align*}
			\sum\limits_{i = 0}^{0} &= 0  \\	
			&= 0 + 0 &\quad(\text{Por aritmética}) \\
			&= 0^2 + 0 &\quad(\text{Por aritmética}) \\
			&= n^2 + n &\quad(\text{Para $n = 0$})
		\end{align*}
	Logo, $P(0)$ é verdadeiro. \\
	\item Para um certo $k \in \mathbb{N}, P(k):= \sum\limits_{i = 0}^{k}i = k^2 + k$ (Hipótese Indutiva). \\
	(Provemos que $P(k + 1) = (k + 1)^2 + (k + 1)$). \\
	\item Então, temos:
		\begin{align*}
			\sum\limits_{i=0}^{k + 1}i &= (\sum\limits_{i = 0}^{k}i) + (k + 1) &\quad(\text{Por def. rec. de Somatório}) \\
			&= (k^2 + k) + (k + 1) &\quad(\text{Por H.I}) \\
			&= k^2 + 2k + 1 \\
			&= (k + 1)^2
		\end{align*}	
\end{enumerate}
