\documentclass[12pt]{article}
\usepackage[utf8]{inputenc}
\usepackage[portuguese]{babel}
\usepackage{amsmath}
\usepackage{amsfonts}
\usepackage{amssymb}
\newtheorem{definition}{Definição}
\newtheorem{postulate}{Postulado}
\newtheorem{theorem}{Teorema}
\newtheorem{corolary}{Corolário}
\newtheorem{principle}{Princípio}
\newtheorem{example}{Exemplo}
\newtheorem{note}{Nota}

\title{Lista de exercícios 01 - FMC}
\author{Andriel Vinicius de M. Fernandes}
\date{\today}

\begin{document}
\maketitle
\section{Indução e Somatório}
\begin{enumerate}
	\item Calcule a soma dos $n$ primeiros pares, em seguida, demonstre por indução. \\
\emph{Resolução.}
Note que:
\begin{align*}
	0             & = 0  \\
	0 + 2         & = 2  \\
	0 + 2 + 4     & = 6  \\
	0 + 2 + 4 + 6 & = 12 \\
\end{align*}
Ou seja, a soma dos $n$ primeiros pares é dada por $n^2 + n$. Provemos essa proposição por indução sobre $n$. \\
Seja $P(n):= \sum\limits_{i = 0}^{n}2i = 0 + 2 + 4 + 6 + \cdots + n = n^2 + n$ \\
Passo base: $n = 0$
\begin{align*}
	\sum\limits_{i = 0}^{0} & = 0                                      \\
	                        & = 0 + 0   & \quad(\text{Por aritmética}) \\
	                        & = 0^2 + 0 & \quad(\text{Por aritmética}) \\
	                        & = n^2 + n & \quad(\text{Para $n = 0$})
\end{align*}
Logo, $P(0)$ é verdadeiro. \\
Passo indutivo: para um certo $k \in \mathbb{N}, P(k):= \sum\limits_{i = 0}^{k}2i = k^2 + k$ (Hipótese Indutiva). \\
(Provemos que $P(k + 1) = (k + 1)^2 + (k + 1)$). \\
Então, temos:
\begin{align*}
	\sum\limits_{i=0}^{k + 1}2i & = (\sum\limits_{i = 0}^{k}2i) + 2(k + 1) & \quad(\text{Por def. rec. de Somatório}) \\
	                            & = (k^2 + k) + 2(k + 1)                   & \quad(\text{Por H.I})                    \\
	                            & = k^2 + k + 2k + 2                       & \quad(\text{Por arit.})                  \\
	                            & = k^2 + 3k + 2                           & \quad(\text{Por arit.})                  \\
	                            & = (k + 1)^2 + (k + 1)                    & \quad(\text{Por arit.})
\end{align*}
Logo, $P(k + 1)$ é verdadeiro e a proposição inicial é verdadeira.

	\item Demonstre que $\forall n \in \mathbb{N}$:
\begin{displaymath}
	\sum\limits_{i=1}^n i^2 = \frac{n (2n+1) (n+1)}{6}
\end{displaymath}
\emph{Resolução.} 
\begin{enumerate}
	\item Indução sobre $n$.
	\item Seja $\forall n \in \mathbb{N}, P(n) := \sum\limits_{i=1}^n i^2 = \frac{n (2n+1) (n+1)}{6}$.
	\item Passo-base: $P(1)$
	\begin{align*}
		\sum\limits_{i=1}^1 i^2 &= 1^2 &\quad(\text{Por def. rec. de $\sum$}) \\
		&= 1 \\
		&= \frac{6}{6} &\quad(\text{Por arit.}) \\
		&= \frac{1 \cdot (2 \cdot 1 + 1) \cdot (1 + 1)}{6} \\
		&= \frac{n \cdot (2n + 1) \cdot (n + 1)}{6} &\quad \text{(Para $n = 1$)}  
	\end{align*}
	Logo, $P(1)$ é verdadeiro.
	\item Para um certo $k \in \mathbb{N}, P(k):= \sum\limits_{i=1}^k i^2 = \frac{k (2k+1) (k+1)}{6}$ (Hipótese Indutiva). \\
	(Provemos que $P(k+1) = \frac{(k+1) (2 \cdot (k+1) + 1) (k + 2)}{6}$). \\
	\item Então, temos:
	\begin{align*}
		P(k+1) &= \sum\limits_{i=1}^{k+1} i^2 \\
		&= (\sum\limits_{i=1}^{k} i^2) + (k+1)^2 &\quad \text{(Por def. rec. de $\sum$)} \\
		&= \frac{k (2k+1) (k+1)}{6} + (k+1)^2 &\quad \text{(Por HI)}\\
		&= \frac{k (2k+1) (k+1) + 6(k+1)^2}{6} &\quad \text{(Por arit.)}\\
		&= \frac{(2k^2 + k) (k+1) + 6k^2 + 12k + 6}{6} &\quad \text{(Por arit.)}\\
		&= \frac{2k^3 + 3k^2 + k + 6k^2 + 12k + 6}{6} &\quad \text{(Por arit.)}\\
		&= \frac{2k^3 + 9k^2 + 13k + 6}{6} &\quad \text{(Por arit.)}\\
		&= \frac{(k + 1) (2 \cdot (k+1) + 1) (k + 2)}{6} &\quad \text{(Por arit.)}\\
	\end{align*}
	Logo, $P(k+1)$ é verdadeiro e a proposição inicial é verdadeira.
\end{enumerate}

	\item Demonstre que $\forall n \in \mathbb{N}$:
\begin{displaymath}
	\sum\limits_{i = 1}^n i(\frac{1}{2})^{i - 1} = 4 - \frac{n + 2}{2^n - 1}
\end{displaymath}
\emph{Resolução}.
\begin{enumerate}
	\item Indução sobre $n$.
	\item Seja $n \in \mathbb{N}, P(n) := \sum\limits_{i = 0}^n i(\frac{1}{2})^{i - 1} = 4 - \frac{n + 2}{2^{n - 1}}$.
	\item Passo-base: $n = 0$:
	\begin{align*}
		\sum\limits_{i = 0}^0 i(\frac{1}{2})^{i - 1} &= 0 \cdot (\frac{1}{2})^{0 - 1} &\quad \text{(Por def. base de $\sum$)} \\
		&= 0 &\quad \text{(Por $x \cdot 0 = 0$)} \\
		&= 4 - 4 &\quad \text{(Por reescrita de 0)} \\
		&= 4 - \frac{2}{\frac{1}{2}} &\quad \text{(Por reescrita de 4)} \\
		&= 4 - \frac{0 + 2}{2^{-1}} &\quad \text{(Por reescrita de $\frac{1}{2}$)} \\
		&= 4 - \frac{n + 2}{2^{n - 1}} &\quad \text{(Para $n = 0$)}
	\end{align*}
	Logo, $P(0)$ é verdadeiro.
	\item Para um certo $k \in \mathbb{N}, P(k) := \sum\limits_{i = 0}^k i(\frac{1}{2})^{i - 1} = 4 - \frac{k + 2}{2^{k - 1}}$ (Hipótese Indutiva). \\
	(Provemos que $P(k + 1) = 4 - \frac{k + 3}{2^{(k + 1) - 1}}$). \\
	\item Então, temos:
	\begin{align*}
		\sum\limits_{i = 0}^{k + 1} i(\frac{1}{2})^{i - 1} &= \sum\limits_{i = 0}^k i(\frac{1}{2})^{i - 1} + (k + 1)\cdot(\frac{1}{2})^k &\quad \text{(Por def. rec. de $\sum$)} \\
		&= 4 - \frac{k + 2}{2^{k - 1}} + (k + 1) \cdot(\frac{1}{2})^k &\quad \text{(Por H.I)} \\
		&= 4 - \frac{k + 2}{2^{k - 1}} + \frac{k + 1}{2^k} &\quad \text{(Por arit.)} \\
		&= 4 + (\frac{-k - 2}{2^{k - 1}}) + \frac{k + 1}{2^k} &\quad \text{(Por arit.)} \\
		&= 4 + (\frac{-k - 2}{2^{k - 1}}) + \frac{\frac{k + 1}{2}}{\frac{2^k}{2}} &\quad \text{(Por divisão de $2$ em $\frac{k + 1}{2^k}$)} \\
		&= 4 + (\frac{-k - 2}{2^{k - 1}}) + \frac{\frac{k + 1}{2}}{2^{k - 1}} &\quad \text{(Por arit.)} \\
		&= 4 + (\frac{-k - 2 + \frac{k + 1}{2}}{2^{k - 1}}) &\quad \text{(Por arit.)} \\
		&= 4 + (\frac{\frac{-2k - 4 + k + 1}{2}}{2^{k - 1}}) &\quad \text{(Por arit.)} \\
		&= 4 + (\frac{\frac{-2k - 4 + k + 1}{2}}{\frac{2^k}{2}}) &\quad \text{(Para $2^{k - 1} = \frac{2^k}{2}$)} \\
		&= 4 + (\frac{-2k - 4 + k + 1}{2^k}) &\quad \text{(Por arit.)} \\
		&= 4 + (\frac{-k - 3}{2^k}) &\quad \text{(Por arit.)} \\
		&= 4 - \frac{k + 3}{2^k} &\quad \text{(Por arit.)} \\
		&= 4 - \frac{k + 3}{2^{k + 1 - 1}} &\quad \text{(Para $2^k = 2^{k + 1 - 1}$)} \\
	\end{align*}
	Logo, $P(k + 1)$ é verdadeiro.
\end{enumerate}

	\item Demonstre que $\forall n \in \mathbb{N}^*$:
\begin{displaymath}
	\sum\limits_{i=1}^n \frac{1}{(2i - 1) (2i + 1)} = \frac{n}{2n + 1}
\end{displaymath}
\emph{Resolução.} \\
Aplicaremos indução sobre $n$. \\
Seja $P(n) := \sum\limits_{i=1}^n \frac{1}{(2i - 1) (2i + 1)} = \frac{n}{2n + 1}$. \\
Passo base: $n = 1$
\begin{align*}
	\sum\limits_{i=1}^1 \frac{1}{(2i - 1) (2i + 1)}
	 & = \frac{1}{(2\cdot1 - 1) (2\cdot1 + 1)} & \quad \text{(Por def. base de $\sum$)} \\
	 & = \frac{1}{1 \cdot 3}                   & \quad \text{(Por arit.)}               \\
	 & = \frac{1}{2 + 1}                       & \quad \text{(Por reescrita de 3)}      \\
	 & = \frac{1}{2 \cdot 1 + 1}               & \quad \text{(Por reescrita de 2)}      \\
	 & = \frac{n}{2 \cdot n + 1}               & \quad \text{(Para $n = 1$)}
\end{align*}
Logo, $P(1)$ é verdadeiro. \\
Passo indutivo: Para um certo $k \in \mathbb{N}^*, P(k) := \sum\limits_{i = 1}^k \frac{1}{(2i - 1) (2i + 1)} = \frac{k}{2k + 1}$ (Hipótese Indutiva). \\
(Provemos que $P(k + 1) = \frac{k + 1}{2k + 3}$.) \\
Então, temos:
\begin{align*}
	\sum\limits_{i=1}^{k + 1} \frac{1}{(2i - 1) (2i + 1)}
	 & =	(\sum\limits_{i=1}^{k} \frac{1}{(2i - 1) (2i + 1)}) + (\frac{1}{(2 \cdot (k + 1) - 1) (2 \cdot (k + 1) + 1)}) \\
	 & =	\frac{k}{2k + 1} + (\frac{1}{(2k + 1) (2k + 3)})                                                              \\
	 & =	\frac{(2k + 3) \cdot k}{(2k + 3) \cdot (2k + 1)} + (\frac{1}{(2k + 1) (2k + 3)})                              \\
	 & =	\frac{2k^2 + 3k + 1}{(2k + 3) \cdot (2k + 1)} + (\frac{1}{(2k + 1) (2k + 3)})                                 \\
	 & = 	\frac{(2k + 1) \cdot (k + 1)}{(2k + 1) (2k + 3)}                                                             \\
	 & = 	\frac{k + 1}{2k + 3}                                                                                         \\
\end{align*}
Logo, $P(k + 1)$ é verdadeiro e a proposição inicial é verdadeira.

\end{enumerate}
\section{Teoria dos Números}
\begin{enumerate}
	\item Demonstre que $\forall x, y, z \in \mathbb{N}$, se $(x + z) = (y + z)$ então $x = y$. \\
\emph{Resolução.} \\
Aplicando indução sobre $z$. \\
Seja $P(n): =\forall x, y \in \mathbb{N}, (x + n) = (y + n) \implies x = y$. \\
Passo base: $n = 0$
\begin{align*}
    P(0):= x + n = y + n & \implies x + 0 = y + 0 & \quad \text{(Para $n = 0$)} \\ & \implies
    x = y                & \quad \text{(Por A1)}
\end{align*}
Logo, $P(0)$ é verdadeiro. \\
Passo indutivo: \\ Seja um certo $k \in \mathbb{N}$ tal que $(x + k) = (y + k) \implies x = y$ (Hipótese Indutiva). \\
(Provemos que $(x + Suc(k)) = (y + Suc(k)) \implies x = y$.) \\
Logo,
\begin{align*}
    P(Suc(k)) := x + Suc(k) = y + Suc(k) & \implies Suc(x + k) = Suc(y + k) & \quad \text{(Por A2)} \\ &\implies
    x + k = y + k                        & \quad \text{(Por S2)}                                    \\ & \implies
    x = y                                & \quad \text{(Por H.I)}
\end{align*}
Logo, $P(Suc(k))$ é verdadeiro e o teorema é válido.
	\item Demonstre que $\forall x, y, z \in \mathbb{N}, x \cdot (y \cdot z) = (x \cdot y) \cdot z$. \\
\emph{Resolução.} \\
Aplicando indução sobre $z$. \\
Seja $P(n) := x \cdot (y \cdot n) = (x \cdot y) \cdot n$. \\
Passo base: $n = 0$
\begin{align*}
    P(0) & = (x \cdot y) \cdot 0                         \\
         & = 0                   & \quad \text{(Por M1)} \\
         & = y \cdot 0           & \quad \text{(Por M1)} \\
         & = x \cdot (y \cdot 0) & \quad \text{(Por M1)} \\
\end{align*}
Logo, $P(0)$ é verdadeiro. \\
Passo indutivo: \\
Seja um certo $k \in \mathbb{N}, x \cdot (y \cdot k) = (x \cdot y) \cdot k$ (Hipótese Indutiva). \\
(Provemos que $P(Suc(k)) := (x \cdot y) \cdot Suc(k) = x \cdot (y \cdot Suc(k)))$. \\
Então, temos:
\begin{align*}
    P(Suc(k)) & = (x \cdot y) \cdot Suc(k)                                           \\
              & =(x \cdot y) \cdot k + (x \cdot y)  & \quad \text{(Por M2)}          \\
              & = x \cdot (y \cdot k) + (x \cdot y) & \quad \text{(Por H.I)}         \\
              & = x \cdot (y \cdot k + y)           & \quad \text{(Pelo teorema Q5)} \\
              & = x \cdot (y \cdot Suc(k))          & \quad \text{(Por M2)}
\end{align*}
Portanto, $P(Suc(k))$ é verdadeiro e o teorema é válido.
	\item Demonstre que $\forall x,y, z \in \mathbb{N}, x \cdot (y + z) = (x \cdot y + x \cdot z)$. \\
\emph{Resolução.}\\
Aplicaremos indução sobre $z$. \\
Seja $P(n) := \forall x, y \in \mathbb{N}, x \cdot (y + n) = (x \cdot y + x \cdot n)$. \\
Passo base: $n = 0$
\begin{align*}
	P(0) & = x \cdot (y + 0)                                   \\
	     & = x \cdot y                 & \quad \text{(Por A1)} \\
	     & = (x \cdot y) + 0           & \quad \text{(Por A1)} \\
	     & = (x \cdot y) + (x \cdot 0) & \quad \text{(Por M1)}
\end{align*}
Logo, $P(0)$ é verdadeiro. \\
Passo indutivo: \\
Seja um certo $k \in \mathbb{N}$ tal que
\begin{displaymath}
	\forall x,y \in \mathbb{N}, x \cdot (y + k) = (x \cdot y) + (x \cdot k) \quad \text{(Hipótese Indutiva)}
\end{displaymath}
(Provemos que $P(Suc(k)):= x \cdot (y + Suc(k)) = (x \cdot y) + (x \cdot Suc(k))$) \\
Logo:
\begin{align*}
	P(Suc(k)) & = x \cdot (y + Suc(k))                                   \\
	          & = x \cdot Suc(y + k)             & \quad \text{(Por A2)} \\
	          & = x \cdot (y + k) + x            & \quad \text{(Por M2)} \\
	          & = (x \cdot y) + (x \cdot k) + x  & \quad \text{(Por HI)} \\
	          & = (x \cdot y) + (x \cdot Suc(k)) & \quad \text{(Por M2)} \\
\end{align*}
Portanto, $P(Suc(k))$ é verdadeiro e o teorema é válido.
\end{enumerate}
\section{Divisibilidade}
\begin{enumerate}
	\item $\forall a, b, c \in \mathbb{N}$ Se $a | b$ e $b | c$, então $a | (b \cdot x + c \cdot y)$ para $\forall x, y \in \mathbb{N}$. \\
\emph{Resolução.} \\
Seja $a, b, c, x, y \in \mathbb{N}$, com $a > 0, b > 0$. \\
Temos que $a | b$, ou seja, $\exists w_1 \in \mathbb{N}, a \cdot w_1 = b$. \\
Temos que $b | c$, ou seja, $\exists w_2 \in \mathbb{N}, b \cdot w_2 = c$. \\
Note que, por aritmética:
\begin{align*}
    a \cdot w_1 = b \iff a \cdot w_1 \cdot x = b \cdot x
\end{align*}
Analogamente, temos:
\begin{align*}
    b \cdot w_2 = c \iff b \cdot w_2 \cdot y = c \cdot y
\end{align*}
Somando as duas equações, teremos:
\begin{align*}
    (a \cdot w_1 \cdot x) + (b \cdot w_2 \cdot y) & = (b \cdot x) + (c \cdot y)         \\ &\implies
    (a \cdot w_1 \cdot x) + (a \cdot w_1 \cdot w_2 \cdot y) = (b \cdot x) + (c \cdot y) \\ &\quad \text{(Por reescrita de $b$)}\\ &\implies
    a \cdot (\cdot w_1 \cdot x + w_1 \cdot w_2 \cdot y) = (b \cdot x) + (c \cdot y)     \\ &\quad \text{(Por aritmética)} \\ &\implies
    a \cdot w_3 = (b \cdot x) + (c \cdot y)                                             \\ &\quad \text{(Para $w_3 = \cdot w_1 \cdot x + w_1 \cdot w_2 \cdot y$)}
\end{align*}
Logo, $\exists w_3 \cdot a = (b \cdot x) + (c \cdot y)$.
Portanto, a proposição é verdadeira.
	\item Demonstre que $\forall a, b, c \in \mathbb{N}, a | b \land a | c \iff a | (b + c)$. \\
\emph{Resolução.} \\
Note que, tomando $a = 2, b = 7, c = 9$, temos que $2 | 16$ (onde $16 = b + c$). \\
Contudo, temos que $2 | 7 \land 2 | 9$ é falso, pois $\lnot \exists w_1 \in \mathbb{N}, 2 \cdot w_1 = 7$ (analogamente para $9$). \\
Portanto, a proposição é falsa.
	\item Demonstre que $\forall a, b \in \mathbb{N}$, Se $a |b$ então $\frac{b}{a} | b$. \\
\emph{Resolução.} \\
Seja $a, b \in \mathbb{N}$, com $a > 0$. \\
Temos que $a | b$, ou seja, $\exists w_1 \in \mathbb{N}$ tal que $a \cdot w_1 = b$. \\
Note que, por aritmética:
\begin{displaymath}
    a \cdot w_1 = b \implies w_1 = \frac{b}{a}
\end{displaymath}
$w_1 \in \mathbb{N}$ por definição. \\
Logo:
\begin{align*}
    a \cdot w_1 = b           & \implies a \cdot \frac{b}{a} = b & \quad \text{(Por reescrita de $w_1$)} \\ & \implies
    w_2 \cdot \frac{b}{a} = b & \quad \text{(Para $w_2 = a$)}
\end{align*}
Logo, $\exists w_2 \in \mathbb{N}$ tal que $w_2 \cdot \frac{b}{a} = b$. \\
Portanto, o teorema é válido.

	\item Demonstre que $\forall n \in \mathbb{N}, 2 | (n^2 + n)$. \\
\emph{Resolução.} \\
Aplicaremos indução sobre $n$. \\
Seja $P(n) := \exists k \in \mathbb{N}, k \cdot 2 = n^2 + n$. \\
Passo base: $n = 0$
\begin{align*}
    2 | (0^2 + 0)
\end{align*}
Note que $2 | 0$ pois, $\forall n \in \mathbb{N}, n | 0$. Logo, $P(0)$ é verdadeiro. \\
Passo indutivo: \\
Seja um certo $k \in \mathbb{N}$. Temos:
\begin{displaymath}
    \exists w_1 \in \mathbb{N}, w_1 \cdot 2 = (k^2 + k)
\end{displaymath}
(Provemos que $P(k + 1) = \exists w_2 \in \mathbb{N}$ tal que $w_2 \cdot 2 = (k + 1)^2 + (k + 1)$).
Logo:
\begin{align*}
    P(k + 1) & = (k + 1)^2 + (k + 1)                                                \\
             & = k^2 + 2k + 1 + k + 1   & \quad \text{(Por aritmética)}             \\
             & = k^2 + k + 2 + 2k       & \quad \text{(Por aritmética)}             \\
             & = (w_1 \cdot 2) + 2 + 2k & \quad \text{(Por HI)}                     \\
             & = 2 \cdot (w_1 + 1 + k)  & \quad \text{(Por aritmética)}             \\
             & = 2 \cdot w_2            & \quad \text{(Para $w_2 = (w_1 + 1 + k)$)} \\
\end{align*}
Logo, $\exists w_2 \in \mathbb{N}$ (naturais são fechados para soma) tal que $w_2 \cdot 2 = (k + 1)^2 + (k + 1)$. \\
Portanto, o teorema é válido.
	\item Demonstre que $\forall n \in \mathbb{N}, 6 | (n^3 - n)$.\\
\emph{Resolução.}\\
Aplicando indução sobre $n$. \\
Seja $P(n) := \exists k \in \mathbb{N}, k \cdot 6 = (n^3 - n)$ \\
Passo base: $n = 0$ \\
Note que $6 | (0^3 - 0)$ pois $\forall n \in \mathbb{N}, n | 0$. \\
Passo indutivo: \\
Seja um certo $k \in \mathbb{N}$. Temos:
\begin{displaymath}
    \exists w_1 \cdot 6 = (k^3 - k) \quad \text{(Hipótese Indutiva)}
\end{displaymath}
(Provemos que $P(k + 1) := \exists w_2 \in \mathbb{N}, w_2 \cdot 6 = (k + 1)^3 - (k + 1)$) \\
Logo:
\begin{align*}
    (k + 1)^3 - (k + 1) & = k^3 + 3k^2 + 3k + 1 - k - 1                  & \quad \text{(Por aritmética)}                                  \\
                        & = k^3 - k + 3k^2 + 3k                          & \quad \text{(Por aritmética)}                                  \\
                        & = (w_1 \cdot 6) + 3k^2 + 3k                    & \quad \text{(Por H.I)}                                         \\
                        & = 6 \cdot (w_1 + \frac{1}{2}k^2 + \frac{1}{2}) & \quad \text{(Por aritmética)}                                  \\
                        & = 6 \cdot w_2                                  & \quad \text{(Para $w_2 = w_1 + \frac{1}{2}k^2 + \frac{1}{2}$)} \\
\end{align*}
Logo, $\exists w_2 \in \mathbb{N}$ (naturais são fechados para soma e multiplicação) tal que $w_2 \cdot 6 = (k + 1)^3 - (k + 1)$. \\
Portanto, o teorema é válido.
	\item Demonstre que para quaisquer inteiros $a, b, c, (mdc(a, c) = 1 \land a | bc) \implies a | b$. \\
\emph{Resolução.} \\
Sejam $a, b, c \in \mathbb{Z}$ arbitrários. \\
Assuma $mdc(a, c) = 1$. Logo, pelo teorema de Bezout, temos que $\exists s, t \in \mathbb{Z}, mdc(a, c) = (a \cdot s) + (c \cdot t).$ \\
Assuma $a | bc$, ou seja, $\exists w_1 \in \mathbb{Z}, a \cdot w_1 = bc$. \\
(Provemos que $\exists w_2 \in \mathbb{Z}, a \cdot w_2 = b$) \\
Note que:
\begin{align*}
	mdc(a, c) = 1 & \implies (a \cdot s) + (c \cdot t) = 1 &\quad \text{(Por reescrita de $mdc$)} \\
	&\implies (a \cdot s) + (c \cdot t) \cdot b = b &\quad \text{(Por multiplicação de $b$)} \\
	&\implies (a \cdot s \cdot b) + (c \cdot t \cdot b) = b &\quad \text{(Por distributividade)} \\
	&\implies (a \cdot s \cdot b) + (t \cdot c \cdot b) = b &\quad \text{(Por comutatividade)} \\
	&\implies (a \cdot s \cdot b) + (t \cdot (w_1 \cdot a)) = b &\quad \text{(Por reescrita de $bc$)} \\
	&\implies a \cdot (s \cdot b) + (t \cdot w_1) = b &\quad \text{(Por reescrita)} \\
	&\implies a \cdot w_2 = b &\quad \text{(Para $w_2 = (s \cdot b) + (t \cdot w_1)$)} \\
\end{align*}
Note que $w_2 \in \mathbb{Z}$ pois as operações realizadas são fechadas sobre os inteiros. \\
Portanto, a proposição é válida.

\end{enumerate}
\end{document}

