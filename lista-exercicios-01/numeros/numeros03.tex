Demonstre que $\forall x, y, z \in \mathbb{N}$, se $(x + z) = (y + z)$ então $x = y$. \\
\emph{Resolução.} \\
Aplicando indução sobre $z$. \\
Seja $P(n): =\forall x, y \in \mathbb{N}, (x + n) = (y + n) \implies x = y$. \\
Passo base: $n = 0$
\begin{align*}
    P(0):= x + n = y + n & \implies x + 0 = y + 0 & \quad \text{(Para $n = 0$)} \\ & \implies
    x = y                & \quad \text{(Por A1)}
\end{align*}
Logo, $P(0)$ é verdadeiro. \\
Passo indutivo: \\ Seja um certo $k \in \mathbb{N}$ tal que $(x + k) = (y + k) \implies x = y$ (Hipótese Indutiva). \\
(Provemos que $(x + Suc(k)) = (y + Suc(k)) \implies x = y$.) \\
Logo,
\begin{align*}
    P(Suc(k)) := x + Suc(k) = y + Suc(k) & \implies Suc(x + k) = Suc(y + k) & \quad \text{(Por A2)} \\ &\implies
    x + k = y + k                        & \quad \text{(Por S2)}                                    \\ & \implies
    x = y                                & \quad \text{(Por H.I)}
\end{align*}
Logo, $P(Suc(k))$ é verdadeiro e o teorema é válido.