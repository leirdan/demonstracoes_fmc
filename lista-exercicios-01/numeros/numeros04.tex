Demonstre que $\forall x, y, z \in \mathbb{N}, x \cdot (y \cdot z) = (x \cdot y) \cdot z$. \\
\emph{Resolução.} \\
Aplicando indução sobre $z$. \\
Seja $P(n) := x \cdot (y \cdot n) = (x \cdot y) \cdot n$. \\
Passo base: $n = 0$
\begin{align*}
    P(0) & = (x \cdot y) \cdot 0                         \\
         & = 0                   & \quad \text{(Por M1)} \\
         & = y \cdot 0           & \quad \text{(Por M1)} \\
         & = x \cdot (y \cdot 0) & \quad \text{(Por M1)} \\
\end{align*}
Logo, $P(0)$ é verdadeiro. \\
Passo indutivo: \\
Seja um certo $k \in \mathbb{N}$ tal que
\begin{displaymath}
    \forall x, y \in \mathbb{N}, x \cdot (y \cdot k) = (x \cdot y) \cdot k \quad \text{(Hipótese Indutiva)}
\end{displaymath}
(Provemos que $P(Suc(k)) := (x \cdot y) \cdot Suc(k) = x \cdot (y \cdot Suc(k)))$ \\
Logo:
\begin{align*}
    P(Suc(k)) & = (x \cdot y) \cdot Suc(k)                                           \\
              & =(x \cdot y) \cdot k + (x \cdot y)  & \quad \text{(Por M2)}          \\
              & = x \cdot (y \cdot k) + (x \cdot y) & \quad \text{(Por H.I)}         \\
              & = x \cdot (y \cdot k + y)           & \quad \text{(Pelo teorema Q5)} \\
              & = x \cdot (y \cdot Suc(k))          & \quad \text{(Por M2)}
\end{align*}
Portanto, $P(Suc(k))$ é verdadeiro e o teorema é válido.